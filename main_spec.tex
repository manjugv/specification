\documentclass[10pt]{book}

\input{utils/packages}

\input{utils/defs}

\makeindex

\begin{document}

\input{content/frontmatter}



\section{The \openshmem Effort}\label{subsec:openshmem_effort}
\input{content/the_openshmem_effort}

\section{Programming Model Overview}\label{subsec:programming_model}
\input{content/programming_model_overview}

\section{Memory Model}\label{subsec:memory_model}
\input{content/memory_model}

\section{Execution Model}\label{subsec:execution_model}
\input{content/execution_model}

\section{Language Bindings and Conformance}\label{subsec:bindings}
\openshmem provides ISO \Cstd and \Fortran[90] language bindings.
Any implementation that provides both \Cstd and \Fortran bindings can claim
conformance to the specification. An implementation that provides e.g.\ only a
\Cstd interface may claim to conform to the \openshmem specification with
respect to the \Cstd language, but not to \Fortran, and should make this
clear in its documentation. The \openshmem header files \HEADER{shmem.h} for
\Cstd and \HEADER{shmem.fh} for
\Fortran must contain only the interfaces and constant names defined in this
specification.

\openshmem \ac{API}s can be implemented as either routines or macros. However,
implementing the interfaces using macros is strongly discouraged as this could
severely limit the use of external profiling tools and high-level compiler
optimizations. An \openshmem program should avoid defining routine names,
variables, or identifiers with the prefix \shmemprefix (for \Cstd and
\Fortran), \shmemprefixC (for \Cstd) or with \openshmem \ac{API} names.

All \openshmem extension \ac{API}s that are not part of this specification must
be defined in the \HEADER{shmemx.h} and \HEADER{shmemx.fh} include files for
\Cstd and \Fortran language bindings, respectively.  These header files
must exist, even if no extensions are provided.  Any extensions shall use the
\FUNC{shmemx\_} prefix for all routine, variable, and constant names.


\section{Library Constants}\label{subsec:library_constants}
\input{content/library_constants}

\section{Environment Variables }\label{subsec:environment_variables}
\input{content/environment_variables}




\clearpage



\section{OpenSHMEM Library API}\label{sec:openshmem_library_api}

\subsection{Library Setup, Exit, and Query Routines}
The library setup and query interfaces that initialize and monitor the parallel
environment of the \ac{PE}s.

\subsubsection{\textbf{SHMEM\_INIT}}\label{subsec:shmem_init}
\input{content/shmem_init}

\subsubsection{\textbf{SHMEM\_MY\_PE}}\label{subsec:shmem_my_pe}
\input{content/shmem_my_pe}

\subsubsection{\textbf{SHMEM\_N\_PES}}\label{subsec:shmem_n_pes}
\input{content/shmem_n_pes}

\subsubsection{\textbf{SHMEM\_FINALIZE}}\label{subsec:shmem_finalize}
\apisummary{
    A collective operation that releases resources used by the \openshmem
    library.  This only terminates the \openshmem portion of a program, not the
    entire program.
}

\begin{apidefinition}

\begin{Csynopsis}
void shmem_finalize(void);
\end{Csynopsis}

\begin{Fsynopsis}
CALL SHMEM_FINALIZE()
\end{Fsynopsis}

\begin{apiarguments}
    \apiargument{None.}{}{}
\end{apiarguments}

\apidescription{
    \FUNC{shmem\_finalize} is a collective operation that ends the \openshmem
    portion of a program previously initialized by \FUNC{shmem\_init} and
    releases resources used by the \openshmem library. This collective
    operation requires all \acp{PE} to participate in the call. There is an
    implicit global barrier in \FUNC{shmem\_finalize} so that pending
    communications are completed, and no resources can be released until all
    \acp{PE} have entered \FUNC{shmem\_finalize}. \FUNC{shmem\_finalize} must be
    the last \openshmem library call encountered in the \openshmem portion of a
    program. A call to \FUNC{shmem\_finalize} will release any resources
    initialized by a corresponding call to \FUNC{shmem\_init}. All processes
    that represent the \acp{PE} will still exist after the
    call to \FUNC{shmem\_finalize} returns, but they will no longer have access
    to any resources that have been released.
}

\apireturnvalues{
    None.
}

\apinotes{
    \FUNC{shmem\_finalize} releases all resources used by the \openshmem library
    including the symmetric memory heap and pointers initiated by
    \FUNC{shmem\_ptr}. This collective operation requires all \acp{PE} to
    participate in the call, not just a subset of the \acp{PE}. The
    non-\openshmem portion of a program may continue after a call to
    \FUNC{shmem\_finalize} by all \acp{PE}. There is an implicit
    \FUNC{shmem\_finalize} at the end of main, so that having an explicit call
    to \FUNC{shmem\_finalize} is optional. However, an explicit
    \FUNC{shmem\_finalize} may be required as an entry point for wrappers used
    by profiling or other tools that need to perform their own finalization.
}

\begin{apiexamples}

\apicexample
    {The following finalize example is for \CorCpp{} programs:}
    {./example_code/shmem_finalize_example.c}
    {}

\end{apiexamples}

\end{apidefinition}


\subsubsection{\textbf{SHMEM\_GLOBAL\_EXIT}}\label{subsec:shmem_global_exit}
\input{content/shmem_global_exit}

\subsubsection{\textbf{SHMEM\_PE\_ACCESSIBLE}}\label{subsec:shmem_pe_accessible}
\apisummary{
    Determines whether a \ac{PE} is accessible via \openshmem's data transfer
    routines.
}

\begin{apidefinition}

\begin{Csynopsis}
int @\FuncDecl{shmem\_pe\_accessible}@(int pe);
\end{Csynopsis}

\begin{apiarguments}
    \apiargument{IN}{pe}{Specific \ac{PE} to be checked for accessibility from
    the local \ac{PE}.}
\end{apiarguments}

\apidescription{
    \FUNC{shmem\_pe\_accessible} is a query routine that indicates whether a
    specified \ac{PE} is accessible via \openshmem from the local \ac{PE}. The
    \FUNC{shmem\_pe\_accessible} routine returns a value indicating whether the remote
    \ac{PE} is a process running from the same executable file as the local
    \ac{PE}, thereby indicating whether full support for symmetric data objects,
    which may reside in either static memory or the symmetric heap, is available.
}

\apireturnvalues{
    The return value is 1 if the specified \ac{PE} is a valid remote \ac{PE}
    for \openshmem routines; otherwise, it is 0.
}

\apinotes{
    None.
}

\end{apidefinition}


\subsubsection{\textbf{SHMEM\_ADDR\_ACCESSIBLE}}\label{subsec:shmem_addr_accessible}
\apisummary{
    Determines whether an address is accessible via \openshmem data transfer
    routines from the specified remote \ac{PE}.
}

\begin{apidefinition}

\begin{Csynopsis}
int @\FuncDecl{shmem\_addr\_accessible}@(const void *addr, int pe);
\end{Csynopsis}

\begin{apiarguments}
    \apiargument{IN}{addr}{Local address of data object to query.}
    \apiargument{IN}{pe}{Integer id of a remote \ac{PE}.}
\end{apiarguments}

\apidescription{
    \FUNC{shmem\_addr\_accessible} is a query routine that indicates whether
    the address \VAR{addr} can be used to access the given data object on the
    specified \ac{PE} via \openshmem routines.
    
    This routine verifies that the data object is symmetric and accessible with
    respect to a remote \ac{PE} via \openshmem data transfer routines.  The
    specified address \VAR{addr} is the local address of the data object on the
    local \ac{PE}.
}

\apireturnvalues{
    The return value is \CONST{1} if the local address \VAR{addr} is also a symmetric
    address and the given data object is accessible via \openshmem routines on
    the specified remote \ac{PE}; otherwise, it is \CONST{0}.
}

\apinotes{
    None.
}

\end{apidefinition}


\subsubsection{\textbf{SHMEM\_PTR}}\label{subsec:shmem_ptr}
\input{content/shmem_ptr}

\subsubsection{\textbf{SHMEM\_INFO\_GET\_VERSION}}\label{subsec:shmem_info_get_version}
\input{content/shmem_info_get_version}

\subsubsection{\textbf{SHMEM\_INFO\_GET\_NAME}}\label{subsec:shmem_info_get_name}
\input{content/shmem_info_get_name}

\subsubsection{\textbf{START\_PES}}\label{subsec:start_pes}
\apisummary{ 
    Called at the beginning of an \openshmem program to initialize the execution
    environment. This routine is deprecated and is provided for backwards
    compatibility. Implementations must include it, and the routine should
    function properly and may notify the user about deprecation of its use.
}

\begin{apidefinition}

\begin{DeprecateBlock}
\begin{Csynopsis}
void start_pes(int npes);
\end{Csynopsis}
\end{DeprecateBlock}

\begin{DeprecateBlock}
\begin{Fsynopsis}
CALL START_PES(npes)
\end{Fsynopsis}
\end{DeprecateBlock}

\begin{apiarguments}
       \apiargument{npes}{Unused}{ Should be set to \CONST{0}.}
\end{apiarguments}

\apidescription{   
     The \FUNC{start\_pes} routine initializes the \openshmem execution
     environment.  An \openshmem program must call \FUNC{start\_pes} before
     calling any other \openshmem routine.
}

\apireturnvalues{
    None.
}

\apinotes{
    If any other \openshmem call occurs before \FUNC{start\_pes}, the
    behavior is undefined.  Although it is recommended to set \VAR{npes} to
    \CONST{0} for \FUNC{start\_pes}, this is not mandated.  The value is ignored.
    Calling \FUNC{start\_pes} more than once has no subsequent
    effect.

    As of \openshmem Specification 1.2 the use of \FUNC{start\_pes} has
    been deprecated. Although \openshmem libraries are required to support the
    call, program users are encouraged to use \FUNC{shmem\_init} instead.
}


\begin{apiexamples}

\apicexample
    { This is a simple program that calls \FUNC{start\_pes}:}
    {./example_code/shmem_startpes_example.f90}
    {} 

\end{apiexamples}

\end{apidefinition}



\subsection{Memory Management Routines}
\label{sec:mem_routines}
\openshmem provides a set of \ac{API}s for managing the symmetric heap. The
\ac{API}s allow one to dynamically allocate, deallocate, reallocate and align
symmetric data objects in the symmetric heap, in \Clang{} and \Fortran.

\subsubsection{\textbf{SHMEM\_MALLOC, SHMEM\_FREE, SHMEM\_REALLOC, SHMEM\_ALIGN}}\label{subsec:shfree}
\apisummary{
    Collective symmetric heap memory management routines.
}

\begin{apidefinition}

\begin{Csynopsis}
void *@\FuncDecl{shmem\_malloc}@(size_t size);
void @\FuncDecl{shmem\_free}@(void *ptr);
void *@\FuncDecl{shmem\_realloc}@(void *ptr, size_t size);
void *@\FuncDecl{shmem\_align}@(size_t alignment, size_t size);
\end{Csynopsis}

\begin{apiarguments}
    \apiargument{IN}{size}{The size, in bytes, of a block to be
        allocated from the symmetric heap.}
    \apiargument{IN}{ptr}{Symmetric address of an object in the symmetric heap.}
    \apiargument{IN}{alignment}{Byte alignment of the block allocated from the
        symmetric heap.}
\end{apiarguments}


\apidescription{
    The \FUNC{shmem\_malloc}, \FUNC{shmem\_free}, \FUNC{shmem\_realloc}, and
    \FUNC{shmem\_align} routines are collective operations that require
    participation by all \acp{PE} in the world team.

    The \FUNC{shmem\_malloc} routine returns the symmetric address of a block of at least
    \VAR{size} bytes, which shall be suitably aligned so that it may be
    assigned to a pointer to any type of object.  This space is allocated from
    the symmetric heap (in contrast to \FUNC{malloc}, which allocates from the
    private heap).  When \VAR{size} is zero, the \FUNC{shmem\_malloc} routine
    performs no action and returns a null pointer.
    
    The \FUNC{shmem\_align} routine allocates a block in the symmetric heap that has
    a byte alignment specified by the \VAR{alignment} argument.  The value of
    \VAR{alignment} shall be a multiple of \CONST{sizeof(void *)} that is also
    a power of two.  Otherwise, the behavior is undefined.  When \VAR{size} is
    zero, the \FUNC{shmem\_align} routine performs no action and returns a null
    pointer.
    
    The \FUNC{shmem\_free} routine causes the block to which \VAR{ptr} points to be
    deallocated, that is, made available for further allocation.  If \VAR{ptr} is a
    null pointer, no action is performed.
           
    The \FUNC{shmem\_realloc} routine changes the size of the block to which
    \VAR{ptr} points to the size (in bytes) specified by \VAR{size}.  The contents
    of the block are unchanged up to the lesser of the new and old sizes.
	The \FUNC{shmem\_realloc} routine preserves allocation hints, e.g. if `ptr`
	was allocated by \FUNC{shmem\_malloc\_with\_hints}.
	If the new size is larger, the newly allocated portion of the block is
    uninitialized.  If \VAR{ptr} is a null pointer, the
    \FUNC{shmem\_realloc} routine behaves like the \FUNC{shmem\_malloc} routine for
    the specified size.  If \VAR{size} is \CONST{0} and \VAR{ptr} is not a
    null pointer, the block to which it points is freed. If the space cannot
    be allocated or if hints cannot be preserved, the block to which \VAR{ptr}
	points is unchanged.
    
    The \FUNC{shmem\_malloc}, \FUNC{shmem\_align}, \FUNC{shmem\_free}, and \FUNC{shmem\_realloc} routines
    are provided  so that multiple \acp{PE} in a program can allocate symmetric,
    remotely accessible memory blocks.  These memory blocks can then be used with
    \openshmem communication routines.  When no action is performed, these
    routines return without performing a \FUNC{shmem\_sync\_all} or
	\FUNC{shmem\_barrier\_all} operation.
	Otherwise, on exit, \FUNC{shmem\_malloc} and \FUNC{shmem\_align} includes at least
	one call to a procedure that is semantically equivalent to
	\FUNC{shmem\_sync\_all}. The \FUNC{shmem\_free} and \FUNC{shmem\_realloc}
	calls a routine that is semantically equivalent to
	\FUNC{shmem\_barrier\_all} both on entry and exit.
    This ensures that all
    \acp{PE} participate in the memory allocation, and that the memory on other
    \acp{PE} can be used as soon as the local \ac{PE} returns.
    The implicit barriers performed by these routines quiet the
    default context.  It is the user's responsibility to ensure that no
    communication operations involving the given memory block are pending on
    other contexts prior to calling
    the \FUNC{shmem\_free} and \FUNC{shmem\_realloc} routines.
    The user is also
    responsible for calling these routines with identical argument(s) on all
    \acp{PE}; if differing \VAR{ptr}, \VAR{size}, or \VAR{alignment} arguments are used, the behavior of the call
    and any subsequent \openshmem calls is undefined.
}

\apireturnvalues{
    The \FUNC{shmem\_malloc} routine returns the symmetric address of the allocated space;
    otherwise, it returns a null pointer.
    
    The \FUNC{shmem\_free} routine returns no value.
    
    The \FUNC{shmem\_realloc} routine returns the symmetric address of the allocated space
    (which may have moved); otherwise, all \acp{PE} return a null pointer.
    
    The \FUNC{shmem\_align} routine returns an aligned symmetric address whose value is a
    multiple of \VAR{alignment}; otherwise, it returns a null pointer.
}

\apinotes{ 
    As of \openshmem[1.2] the use of \FUNC{shmalloc}, \FUNC{shmemalign},
    \FUNC{shfree},  and \FUNC{shrealloc} has been deprecated. Although \openshmem
    libraries are required to support the calls, users are encouraged to use
    \FUNC{shmem\_malloc}, \FUNC{shmem\_align}, \FUNC{shmem\_free}, and
    \FUNC{shmem\_realloc} instead.  The behavior and signature  of the routines
    remains unchanged from the deprecated versions.
    					 
    The total size of the symmetric heap is determined at job startup.  One can
    specify the size of the heap using the \ENVVAR{SHMEM\_SYMMETRIC\_SIZE} environment
    variable (where available).	
    
    The \FUNC{shmem\_malloc}, \FUNC{shmem\_free}, and \FUNC{shmem\_realloc} routines
    differ from the private heap allocation routines in that all \acp{PE} in a
    program must call them (a barrier is used to ensure this).

    When the \VAR{ptr} argument in a call to \FUNC{shmem\_realloc} corresponds
    to a buffer allocated using \FUNC{shmem\_align}, the buffer returned by
    \FUNC{shmem\_realloc} is not guaranteed to maintain the alignment requested
    in the original call to \FUNC{shmem\_align}.
}

\apiimpnotes{
    The symmetric heap allocation routines always return the symmetric addresses of corresponding
    symmetric objects across all \acp{PE}. The \openshmem specification does not
    require that the virtual addresses are equal across all \acp{PE}. Nevertheless,
    the implementation must avoid costly address translation operations in the
    communication path, including $O(N)$ memory translation tables,
    where $N$ is the number of \acp{PE}.  In order to avoid address translations, the
    implementation may re-map the allocated block of memory based on agreed virtual
    address.  Additionally, some operating systems provide an option to disable
    virtual address randomization, which enables predictable allocation of virtual
    memory addresses.
}

\end{apidefinition}


\subsubsection{\textbf{SHPALLOC}}\label{subsec:shpalloc}
\input{content/shpalloc.tex}

\subsubsection{\textbf{SHPCLMOVE}}\label{subsec:shpclmove}
\input{content/shpclmove.tex}

\subsubsection{\textbf{SHPDEALLOC}}\label{subsec:shpdealloc}
\input{content/shpdealloc.tex}

\color{ForestGreen}
\subsection{OpenSHMEM and Threads}
This section specifies the interaction between \openshmem{} interfaces and the
user threads, and also describes the routines that can be used for initializing and 
querying the thread environment.
 

\begin{itemize}

\item
In a thread-compliant implementation, an \openshmem{} PE is an OS process that
is multithreaded. Each thread can issue \openshmem{} calls; however the threads
are not separately addressable. The symmetric heap and symmetric variables scope
are not impacted by multiple threads invoking the
\openshmem{} interfaces, i.e., the symmetric heap is a per-process resource.
For example, a thread invoking a memory allocation routine such as SHMEM\_MALLOC() 
affects the entire process. The requirement that the same symmetric heap operations must
be executed by all processes in the same order also applies in a threaded
environment. 
                                    	
\item In a thread-compliant implementation, 
all \openshmem{} calls are thread-safe, i.e., two concurrently running threads
may make \openshmem{} calls and the outcome will be as if the calls executed in
some order, even if their execution is interleaved.

\item Blocking \openshmem{} calls will block the calling thread only, allowing another
thread to execute, if available. The calling thread will be blocked until the
event on which it is waiting occurs. Once the blocked communication is enabled
and can proceed, then the call will complete and the thread will be marked
runnable, within a finite time. A blocked thread will not prevent progress of
other runnable threads on the same \ac{PE}, and will not prevent them from
executing other \openshmem{} calls. Also, a blocked thread will not prevent the
progress of other \openshmem{} calls on other \acp{PE}. 
 
\item
In a thread-compliant implementation, if multiple threads call the collective
calls, it is the programmer's responsibility to ensure the correct ordering of
collective calls.  The symmetric heap management functions, which are defined to call
SHMEM\_BARRIER\_ALL()(\ref{sec:mem_routines}) before they return 
must be treated as collective operations.

\item
\openshmem{} thread calling SHMEM\_INIT() is designated as the main thread.
Multiple threads may not call SHMEM\_INIT(). Similarly, \openshmem{} finalize
may only be called on the main thread.

\end{itemize} 
 
{\bf Clarifications:}
 
\begin{itemize}
\item[]
The \openshmem{} specification can be implemented without support for threads.
The \openshmem{} implementations are not required to be thread complaint.
Regardless of whether \openshmem{} is thread complaint or not, SHMEM\_INIT(),
SHMEM\_FINALIZE(), SHMEM\_GLOBALEXIT(), SHMEM\_INFO\_GET\_NAME(), and
SHMEM\_INFO\_GET\_VERSION() should be thread safe.
 
%\item[]
%Interaction with signals: The outcome is undefined if a thread that executes an
%\openshmem{} call catches a signal. However, a thread of an \openshmem{} process
%may terminate, and may catch signals or be canceled by another thread when not
%executing \openshmem{} calls.

\item[]
The completion semantics are not impacted by the multiple threads. 
For example, the shmem\_barrier\_all() is completed when all \acp{PE} enter and
exit the shmem\_barrier\_all() call, even though only one thread in the \ac{PE} is
participating in the collective call. 

 
\end{itemize}


\subsubsection{\textbf{SHMEM\_INIT\_THREAD}}
\apisummary{
Initializes the \openshmem{} library, similar to SHMEM\_INIT(), and in addition
perform the initialization required for thread-safe invocation of \openshmem{} functions.}

\begin{apidefinition}

\begin{Csynopsis}
int shmem_init_thread(int requested, int *provided);
\end{Csynopsis}

\begin{apiarguments}
\apiargument{IN}{requested}  {The thread level support requested by the user.
The correct values are SHMEM\_THREAD\_SINGLE, SHMEM\_THREAD\_FUNNELED, or SHMEM\_THREAD\_MULTIPLE}
\apiargument{OUT}{provided}{The thread level support provided by the \openshmem{} implementation.}
\end{apiarguments}


\apidescription{
    \FUNC{shmem\_init\_thread}  initializes the \openshmem{} library similar to
    \FUNC{shmem\_init}, and in addition perform the initialization required for
    thread-safe invocation of \openshmem{} functions. The argument
    \textit{requested} is used to specify the desired level of thread support.
    The function returns the support provided by the library. There are four 
    levels of threading support.
 
 \begin{itemize}
\item SHMEM\_THREAD\_SINGLE: The \openshmem{} program may not be multithreaded, 
and all \openshmem{} interfaces are invoked by a single thread. 

\item SHMEM\_THREAD\_FUNNELED: 
The \openshmem{} program may be multithreaded, however, the 
program must ensure that only one thread invokes the \openshmem{}
interfaces.

\item SHMEM\_THREAD\_SERIALIZED: 
The \openshmem{} program may be multithreaded, however, the 
program must ensure that only one thread invokes the \openshmem{}
interfaces at any given instance i.e., the \openshmem{} interfaces 
are not invoked concurrently by multiple threads.

\item SHMEM\_THREAD\_MULTIPLE: 
The \openshmem{} program may be multithreaded, and any 
thread may invoke the \openshmem{} interfaces.

\item SHMEM\_THREAD\_UNSUPPORTED: The \openshmem{} specification 
can be implemented without the support for threads, or a model beyond the four
models.
   
\end{itemize}

The function may be used to initialize \openshmem{}, and to initialize the
\openshmem{} with thread safety, instead of SHMEM\_INIT. The SHMEM\_INIT\_THREAD
may not be called multiple times in an \openshmem{} program.
}

\apireturnvalues{    
	NONE
    }

\apinotes{
The \openshmem{} programming model does not recognize individual threads.  Any
\openshmem{} operation initiated by a thread is considered an action of the
process as a whole. Thread-safety should not be activated unless needed.
Activating  thread-safety causes additional overhead even if no additional
threads are created or used.
}		


\end{apidefinition}

 
  

\subsubsection{\textbf{SHMEM\_QUERY\_THREAD}}
\apisummary{
Returns the level of thread support provided by the library.}


\begin{apidefinition}

\begin{Csynopsis}
void shmem_query_thread(int *provided);
\end{Csynopsis}

\begin{apiarguments}
\apiargument{OUT}{provided}{The thread level support provided by the \openshmem{} implementation.}
\end{apiarguments}


\apidescription{
The \FUNC{shmem\_query\_thread} call returns the level of thread support
currently being provided. The value returned will be same as \VAR{provided}
returned in the \FUNC{shmem\_init\_thread()}, if the \openshmem{} library was
initialized by \FUNC{shmem\_init\_thread()}.
}

\apireturnvalues{    
	NONE
}

\apinotes{
}

\end{apidefinition}

 
  

\color{Black}

\subsection{Remote Memory Access Routines}\label{sec:rma}
\input{content/rma_intro.tex}

\subsubsection{\textbf{SHMEM\_PUT}}\label{subsec:shmem_put}
\apisummary{
    The  put routines  provide  a method for copying data from a contiguous local
    data object to a data object on a specified \ac{PE}.
}

\begin{apidefinition}

\begin{C11synopsis}
void @\FuncDecl{shmem\_put}@(TYPE *dest, const TYPE *source, size_t nelems, int pe);
void @\FuncDecl{shmem\_put}@(shmem_ctx_t ctx, TYPE *dest, const TYPE *source, size_t nelems, int pe);
\end{C11synopsis}
where \TYPE{} is one of the standard \ac{RMA} types specified by Table \ref{stdrmatypes}.

\begin{Csynopsis}
void @\FuncDecl{shmem\_\FuncParam{TYPENAME}\_put}@(TYPE *dest, const TYPE *source, size_t nelems, int pe);
void @\FuncDecl{shmem\_ctx\_\FuncParam{TYPENAME}\_put}@(shmem_ctx_t ctx, TYPE *dest, const TYPE *source, size_t nelems, int pe);
\end{Csynopsis}
where \TYPE{} is one of the standard \ac{RMA} types and has a corresponding \TYPENAME{} specified by Table \ref{stdrmatypes}.

\begin{CsynopsisCol}
void @\FuncDecl{shmem\_put\FuncParam{SIZE}}@(void *dest, const void *source, size_t nelems, int pe);
void @\FuncDecl{shmem\_ctx\_put\FuncParam{SIZE}}@(shmem_ctx_t ctx, void *dest, const void *source, size_t nelems, int pe);
\end{CsynopsisCol}
where \SIZE{} is one of \CONST{8, 16, 32, 64, 128}.

\begin{CsynopsisCol}
void @\FuncDecl{shmem\_putmem}@(void *dest, const void *source, size_t nelems, int pe);
void @\FuncDecl{shmem\_ctx\_putmem}@(shmem_ctx_t ctx, void *dest, const void *source, size_t nelems, int pe);
\end{CsynopsisCol}

\begin{Fsynopsis}
CALL @\FuncDecl{SHMEM\_CHARACTER\_PUT}@(dest, source, nelems, pe)
CALL @\FuncDecl{SHMEM\_COMPLEX\_PUT}@(dest, source, nelems, pe)
CALL @\FuncDecl{SHMEM\_DOUBLE\_PUT}@(dest, source, nelems, pe)
CALL @\FuncDecl{SHMEM\_INTEGER\_PUT}@(dest, source, nelems, pe)
CALL @\FuncDecl{SHMEM\_LOGICAL\_PUT}@(dest, source, nelems, pe)
CALL @\FuncDecl{SHMEM\_PUT4}@(dest, source, nelems, pe)
CALL @\FuncDecl{SHMEM\_PUT8}@(dest, source, nelems, pe)
CALL @\FuncDecl{SHMEM\_PUT32}@(dest, source, nelems, pe)
CALL @\FuncDecl{SHMEM\_PUT64}@(dest, source, nelems, pe)
CALL @\FuncDecl{SHMEM\_PUT128}@(dest, source, nelems, pe)
CALL @\FuncDecl{SHMEM\_PUTMEM}@(dest, source, nelems, pe)
CALL @\FuncDecl{SHMEM\_REAL\_PUT}@(dest, source, nelems, pe)
\end{Fsynopsis}

\begin{apiarguments}
    \apiargument{IN}{ctx}{A context handle specifying the context on which to perform the operation.
      When this argument is not provided, the operation is performed on
      the default context.}
    \apiargument{OUT}{dest}{Data object to be updated on the remote \ac{PE}. This
    data object must be remotely accessible.}
    \apiargument{IN}{source}{Data object containing the data to be copied.}
    \apiargument{IN}{nelems}{Number of elements in the \VAR{dest} and \VAR{source}
    arrays. \VAR{nelems} must be of type \VAR{size\_t} for \Cstd. When using
    \Fortran, it must be a constant, variable, or array element of default
    integer type.}
    \apiargument{IN}{pe}{\ac{PE} number of the remote \ac{PE}. \VAR{pe} must be
    of type integer. When using \Fortran, it must be a constant, variable,
    or array element of default integer type.}
\end{apiarguments}

\apidescription{
    The routines return after the data has been copied out of the \source{} array
    on the local \ac{PE}.  The delivery of data words into the data object on the
    destination \ac{PE} may occur in any order.  Furthermore, two successive put
    routines may deliver data out of order unless a call to \FUNC{shmem\_fence} is
    introduced between the two calls. The routines are considered complete after
	a subsequent call to \FUNC{shmem\_quiet}.
    If the context handle \VAR{ctx} does not correspond to a valid context,
    the behavior is undefined.
 }

\apidesctable{
    The \dest{} and \source{} data objects must conform to certain typing
    constraints, which are as follows:}
    {Routine}{Data type of \VAR{dest} and \VAR{source}}
    \apitablerow{shmem\_putmem}{\Fortran: Any noncharacter type. \Cstd: Any
        data  type.  nelems is scaled in bytes.}
    \apitablerow{shmem\_put4, shmem\_put32}{Any noncharacter type
        that has a storage size equal to \CONST{32} bits.}
    \apitablerow{shmem\_put8}{\Cstd: Any noncharacter type that
        has a storage size equal to \CONST{8} bits.}
    \apitablerow{}{\Fortran: Any noncharacter type that
        has a storage size equal to \CONST{64} bits.}
    \apitablerow{shmem\_put64}{Any noncharacter type that
        has a storage size equal to \CONST{64} bits.}
    \apitablerow{shmem\_put128}{Any noncharacter type that has a
        storage size equal to \CONST{128} bits.}
    \apitablerow{SHMEM\_CHARACTER\_PUT}{Elements of type character.  \VAR{nelems}
    is  the number  of	 characters to transfer. The actual character lengths of
    the \source{} and \dest{} variables are ignored. }
    \apitablerow{SHMEM\_COMPLEX\_PUT}{Elements of type complex of default size.}
    \apitablerow{SHMEM\_DOUBLE\_PUT}{Elements of type double precision. }
    \apitablerow{SHMEM\_INTEGER\_PUT}{Elements of type integer.}
    \apitablerow{SHMEM\_LOGICAL\_PUT}{Elements of type logical.}
    \apitablerow{SHMEM\_REAL\_PUT}{Elements of type real.}

\apireturnvalues{
    None.
}
\apinotes{
    When using \Fortran, data types must be of default size.  For example,
    a real variable must be declared as \CONST{REAL},  \CONST{REAL*4},  or
    \CONST{REAL(KIND=KIND(1.0))}.
    As of \openshmem[1.2], the \Fortran API routine \FUNC{SHMEM\_PUT} has
    been deprecated, and either \FUNC{SHMEM\_PUT8} or \FUNC{SHMEM\_PUT64} should
    be used in its place.
}

\begin{apiexamples}

\apicexample
    { The following \FUNC{shmem\_put} example is for \Cstd[11] programs:}
    {./example_code/shmem_put_example.c}
    {} 
\end{apiexamples}

\end{apidefinition}


\subsubsection{\textbf{SHMEM\_P}}\label{subsec:shmem_p}
\input{content/shmem_p.tex}

\subsubsection{\textbf{SHMEM\_IPUT}}\label{subsec:shmem_iput}
\apisummary{
    Copies strided data to a specified \ac{PE}.
}

\begin{apidefinition}

\begin{C11synopsis}
void @\FuncDecl{shmem\_iput}@(TYPE *dest, const TYPE *source, ptrdiff_t dst, ptrdiff_t sst, size_t nelems, int pe);
void @\FuncDecl{shmem\_iput}@(shmem_ctx_t ctx, TYPE *dest, const TYPE *source, ptrdiff_t dst, ptrdiff_t sst, size_t nelems, int pe);
\end{C11synopsis}
where \TYPE{} is one of the standard \ac{RMA} types specified by Table \ref{stdrmatypes}.

\begin{Csynopsis}
void @\FuncDecl{shmem\_\FuncParam{TYPENAME}\_iput}@(TYPE *dest, const TYPE *source, ptrdiff_t dst, ptrdiff_t sst, size_t nelems, int pe);
void @\FuncDecl{shmem\_ctx\_\FuncParam{TYPENAME}\_iput}@(shmem_ctx_t ctx, TYPE *dest, const TYPE *source, ptrdiff_t dst, ptrdiff_t sst, size_t nelems, int pe);
\end{Csynopsis}
where \TYPE{} is one of the standard \ac{RMA} types and has a corresponding \TYPENAME{} specified by Table \ref{stdrmatypes}.

\begin{CsynopsisCol}
void @\FuncDecl{shmem\_iput\FuncParam{SIZE}}@(void *dest, const void *source, ptrdiff_t dst, ptrdiff_t sst, size_t nelems, int pe);
void @\FuncDecl{shmem\_ctx\_iput\FuncParam{SIZE}}@(shmem_ctx_t ctx, void *dest, const void *source, ptrdiff_t dst, ptrdiff_t sst, size_t nelems, int pe);
\end{CsynopsisCol}
where \SIZE{} is one of \CONST{8, 16, 32, 64, 128}.

\begin{Fsynopsis}
INTEGER dst, sst, nelems, pe
CALL @\FuncDecl{SHMEM\_COMPLEX\_IPUT}@(dest, source, dst, sst, nelems, pe)
CALL @\FuncDecl{SHMEM\_DOUBLE\_IPUT}@(dest, source, dst, sst, nelems, pe)
CALL @\FuncDecl{SHMEM\_INTEGER\_IPUT}@(dest, source, dst, sst, nelems, pe)
CALL @\FuncDecl{SHMEM\_IPUT4}@(dest, source, dst, sst, nelems, pe)
CALL @\FuncDecl{SHMEM\_IPUT8}@(dest, source, dst, sst, nelems, pe)
CALL @\FuncDecl{SHMEM\_IPUT32}@(dest, source, dst, sst, nelems, pe)
CALL @\FuncDecl{SHMEM\_IPUT64}@(dest, source, dst, sst, nelems, pe)
CALL @\FuncDecl{SHMEM\_IPUT128}@(dest, source, dst, sst, nelems, pe)
CALL @\FuncDecl{SHMEM\_LOGICAL\_IPUT}@(dest, source, dst, sst, nelems, pe)
CALL @\FuncDecl{SHMEM\_REAL\_IPUT}@(dest, source, dst, sst, nelems, pe)
\end{Fsynopsis}

\begin{apiarguments}
    \apiargument{IN}{ctx}{A context handle specifying the context on which to perform the operation.
        When this argument is not provided, the operation is performed on
        the default context.}
    \apiargument{OUT}{dest}{Array to be updated on the remote \ac{PE}. This data
        object  must be remotely accessible.}
    \apiargument{IN}{source}{Array containing the data to be copied.}
    \apiargument{IN}{dst}{The stride between consecutive elements of the \dest{}
        array.  The stride is scaled by the element size of the \dest{} array.  A
        value of \CONST{1} indicates contiguous data.  \VAR{dst} must be of type
        \CTYPE{ptrdiff\_t}.  When using \Fortran, it must be a default integer value.}
    \apiargument{IN}{sst}{The  stride between consecutive elements of the
        \source{} array.  The stride is scaled by the element size of the \source{}
        array.  A  value of \CONST{1} indicates contiguous data.  \VAR{sst} must be
        of type \CTYPE{ptrdiff\_t}.  When using \Fortran, it must be a
        default integer value.}
    \apiargument{IN}{nelems}{Number of elements in the \dest{} and \source{}
        arrays.  \VAR{nelems} must be of type \VAR{size\_t} for \Cstd.  When
        using \Fortran, it must be  a constant, variable, or array element of
        default integer type.}
    \apiargument{IN}{pe}{\ac{PE} number of the remote \ac{PE}.  \VAR{pe} must be
        of type integer.   When using  \Fortran, it must be a constant,
        variable, or array element of default integer type.}
\end{apiarguments}


\apidescription{
    The \FUNC{iput} routines provide a method  for  copying strided data
    elements (specified by \VAR{sst}) of an array from a \source{} array on the
    local \ac{PE} to locations specified by stride \VAR{dst} on a \dest{} array
    on specified remote \ac{PE}. Both strides, \VAR{dst} and \VAR{sst}, must be
    greater than or equal to \CONST{1}. The routines return when the data has
    been copied out of the \VAR{source} array on the local \ac{PE} but not
    necessarily before the data has been delivered to the remote data object.
	The routines are considered complete after a subsequent call to
	\FUNC{shmem\_quiet}.
	If the context handle \VAR{ctx} does not correspond to a valid context,
     the behavior is undefined.
}

\apidesctable{
    The \dest{} and \source{} data objects must conform to typing constraints,
    which are as follows:
}{Routine}{Data type of \VAR{dest} and \VAR{source}}
    \apitablerow{shmem\_iput4, shmem\_iput32}{Any noncharacter type
        that has a storage size equal to \CONST{32} bits.}
    \apitablerow{shmem\_iput8}{\Cstd: Any noncharacter type that
        has a storage size equal to \CONST{8} bits.}
    \apitablerow{}{\Fortran: Any noncharacter type that
        has a storage size equal to \CONST{64} bits.}
    \apitablerow{shmem\_iput64}{Any noncharacter type that
        has a storage size equal to \CONST{64} bits.}
    \apitablerow{shmem\_iput128}{Any noncharacter type that has a
        storage size equal to \CONST{128} bits.}
    \apitablerow{SHMEM\_COMPLEX\_IPUT}{Elements of type complex of default size.}
    \apitablerow{SHMEM\_DOUBLE\_IPUT}{Elements of type double precision.}
    \apitablerow{SHMEM\_INTEGER\_IPUT}{Elements of type integer.}
    \apitablerow{SHMEM\_LOGICAL\_IPUT}{Elements of type logical.}
    \apitablerow{SHMEM\_REAL\_IPUT}{Elements of type real.}

\apireturnvalues{
    None.
}

\apinotes{
    When using \Fortran, data types must be of default size.  For example, a
    real variable must be declared as  \CONST{REAL}, \CONST{REAL*4} or
    \CONST{REAL(KIND=KIND(1.0))}.
    See Section \ref{subsec:memory_model} for a definition of the term
    remotely accessible.
}

\begin{apiexamples}

\apicexample
    {Consider the following \FUNC{shmem\_iput} example for \Cstd[11] programs.}
    {./example_code/shmem_iput_example.c}
    {}
\end{apiexamples}

\end{apidefinition}


\subsubsection{\textbf{SHMEM\_GET}}\label{subsec:shmem_get}
\input{content/shmem_get.tex}

\subsubsection{\textbf{SHMEM\_G}}\label{subsec:shmem_g}
\input{content/shmem_g.tex}

\subsubsection{\textbf{SHMEM\_IGET}}\label{subsec:shmem_iget}
\input{content/shmem_iget.tex}


\subsection{Non-blocking Remote Memory Access Routines}\label{sec:rma_nbi}

\subsubsection{\textbf{SHMEM\_PUT\_NBI}}\label{subsec:shmem_put_nbi}
\input{content/shmem_put_nbi.tex}

\subsubsection{\textbf{SHMEM\_GET\_NBI}}\label{subsec:shmem_get_nbi}
\input{content/shmem_get_nbi.tex}


\subsection{Atomic Memory Operations}\label{sec:amo}
\input{content/atomics_intro}

\subsubsection{\textbf{SHMEM\_ADD}}\label{subsec:shmem_add}
\input{content/shmem_add.tex}

\subsubsection{\textbf{SHMEM\_CSWAP}}\label{subsec:shmem_cswap}
\input{content/shmem_cswap.tex} 

\subsubsection{\textbf{SHMEM\_SWAP}}\label{subsec:shmem_swap}
\input{content/shmem_swap.tex}

\subsubsection{\textbf{SHMEM\_FINC}}\label{subsec:shmem_finc}
\input{content/shmem_finc.tex}

\subsubsection{\textbf{SHMEM\_INC}}\label{subsec:shmem_inc}
\input{content/shmem_inc.tex}

\subsubsection{\textbf{SHMEM\_FADD}}\label{subsec:shmem_fadd}
\input{content/shmem_fadd.tex}

\subsubsection{\textbf{SHMEM\_FETCH}}\label{subsec:shmem_fetch}
\input{content/shmem_fetch.tex}

\subsubsection{\textbf{SHMEM\_SET}}\label{subsec:shmem_set}
\input{content/shmem_set.tex}





\subsection{Collective Routines}\label{subsec:coll}
\input{content/collective_intro.tex}

\subsubsection{\textbf{SHMEM\_BARRIER\_ALL}}\label{subsec:shmem_barrier_all}
\input{content/shmem_barrier_all.tex}

\subsubsection{\textbf{SHMEM\_BARRIER}}\label{subsec:shmem_barrier}
\input{content/shmem_barrier.tex}

\subsubsection{\textbf{SHMEM\_BROADCAST}}\label{subsec:shmem_broadcast}
\input{content/shmem_broadcast.tex}

\subsubsection{\textbf{SHMEM\_COLLECT, SHMEM\_FCOLLECT}}\label{subsec:shmem_collect}
\input{content/shmem_collect.tex}

\subsubsection{\textbf{SHMEM\_REDUCTIONS}}\label{subsec:shmem_reductions}
\input{content/shmem_reductions.tex}

\subsubsection{\textbf{SHMEM\_ALLTOALL}}\label{subsec:shmem_alltoall}
\apisummary{
    shmem\_alltoall is a collective routine where each \ac{PE} exchanges a fixed amount of data with all other \acp{PE} in the
    active set.
}

\begin{apidefinition}

\begin{Csynopsis}
void @\FuncDecl{shmem\_alltoall32}@(void *dest, const void *source, size_t nelems, int PE_start, int logPE_stride, int PE_size, long *pSync);
void @\FuncDecl{shmem\_alltoall64}@(void *dest, const void *source, size_t nelems, int PE_start, int logPE_stride, int PE_size, long *pSync);
\end{Csynopsis}

\begin{Fsynopsis}
INTEGER pSync(SHMEM_ALLTOALL_SYNC_SIZE)
INTEGER PE_start, logPE_stride, PE_size, nelems
CALL @\FuncDecl{SHMEM\_ALLTOALL32}@(dest, source, nelems, PE_start, logPE_stride, PE_size, pSync)
CALL @\FuncDecl{SHMEM\_ALLTOALL64}@(dest, source, nelems, PE_start, logPE_stride, PE_size, pSync)
\end{Fsynopsis}

\begin{apiarguments}

\apiargument{OUT}{dest}{A symmetric data object large enough to receive
    the combined total of \VAR{nelems} elements from each \ac{PE} in the
    active set.}
\apiargument{IN}{source}{A symmetric data object that contains \VAR{nelems}
    elements of data for each \ac{PE} in the active set, ordered according to
    destination \ac{PE}.}
\apiargument{IN}{nelems}{The number of elements to exchange for each \ac{PE}.
    \VAR{nelems} must be of type size\_t for \CorCpp.  When using
    \Fortran, it must be a default integer value.}
\apiargument{IN}{PE\_start}{The lowest \ac{PE} number of the active set of
    \acp{PE}.  \VAR{PE\_start} must be of type integer.  When using \Fortran,
    it must be a default integer value.}
\apiargument{IN}{logPE\_stride}{The log (base 2) of the stride between
    consecutive \ac{PE} numbers in the active set.  \VAR{logPE\_stride} must be of
    type integer.  When using \Fortran, it must be a default integer value.}
\apiargument{IN}{PE\_size}{The number of \acp{PE} in the active set.
    \VAR{PE\_size} must be of type integer.  When using \Fortran, it must
    be a default integer value.}
\apiargument{IN}{pSync}{
    A symmetric work array of size \CONST{SHMEM\_ALLTOALL\_SYNC\_SIZE}.
    In \CorCpp, \VAR{pSync} must be an array of elements of type \CTYPE{long}.
    In \Fortran, \VAR{pSync} must be an array of elements of default integer type.
    Every element of this array must be initialized with the value
    \CONST{SHMEM\_SYNC\_VALUE} before any of the \acp{PE} in the active set
    enter the routine.}

\end{apiarguments}

\apidescription{
    The \FUNC{shmem\_alltoall} routines are collective routines. Each \ac{PE}
    in the active set exchanges \VAR{nelems} data elements of size
    32 bits (for \FUNC{shmem\_alltoall32}) or 64 bits (for \FUNC{shmem\_alltoall64})
    with all other \acp{PE} in the set. The data being sent and received are
    stored in a contiguous symmetric data object. The total size of each \acp{PE}
    \VAR{source} object and \VAR{dest} object is \VAR{nelems} times the size of
    an element (32 bits or 64 bits) times \VAR{PE\_size}.
    The \VAR{source} object contains \VAR{PE\_size} blocks of data (the size of each
    block defined by \VAR{nelems}) and each block of data is sent to a different \ac{PE}. 
    Given a \ac{PE} \VAR{i} that is the \kth PE in the active set and a \ac{PE}
    \VAR{j} that is the \lth \ac{PE} in the active set,
    \ac{PE} \VAR{i} sends the \lth block of its \VAR{source} object to
    the \kth block of
    the \VAR{dest} object of \ac{PE} \VAR{j}.

    As with all \openshmem collective routines, this routine assumes
    that only \acp{PE} in the active set call the routine.  If a \ac{PE} not
    in the active set calls an \openshmem collective routine,
    the behavior is undefined.

    The values of arguments \VAR{nelems}, \VAR{PE\_start}, \VAR{logPE\_stride},
    and \VAR{PE\_size} must be equal on all \acp{PE} in the active set. The same
    \VAR{dest} and \VAR{source} data objects, and the same \VAR{pSync} work
    array must be passed to all \acp{PE} in the active set.

    Before any \ac{PE} calls a \FUNC{shmem\_alltoall} routine,
    the following conditions must be ensured:
    \begin{itemize}
    
    \item \oldtext{The \VAR{pSync} array on all \acp{PE} in the active set is not
      still in use from a prior call to a \FUNC{shmem\_alltoall} routine.
    }
    \item \newtext{The \VAR{pSync} array on all \acp{PE} in the active set should
    not be in use from a prior call to a \FUNC{shmem\_alltoall} routine
    (if the pSync array specific to \FUNC{shmem\_alltoall} is used)
    or any of the collective routines (if the same pSync array is
    used for all collective operations).
    }
    \item The \VAR{dest} data object on all \acp{PE} in the active set is
      ready to accept the \FUNC{shmem\_alltoall} data.
    \end{itemize}
    Otherwise, the behavior is undefined.

    Upon return from a \FUNC{shmem\_alltoall} routine, the following is true for
    the local PE: Its \VAR{dest} symmetric data object is completely updated and
    the data has been copied out of the \VAR{source} data object.
    The values in the \VAR{pSync} array are restored to the original values.
}

\apidesctable{
The  \dest{}  and \source{} data  objects must conform to certain typing
constraints, which are as follows:
}{Routine}{Data type of \VAR{dest} and \VAR{source}}

\apitablerow{shmem\_alltoall64}{\CONST{64} bits aligned.}
\apitablerow{shmem\_alltoall32}{\CONST{32} bits aligned.}

\apireturnvalues{
    None.
}

\apinotes{
    This routine restores \VAR{pSync} to its original contents.  Multiple calls
    to \openshmem\ routines that use the same \VAR{pSync} array do not require
    that \VAR{pSync} be reinitialized after the first call.
    The user must ensure that the \VAR{pSync} array is not being updated by any
    \ac{PE} in the active set while any of the \acp{PE} participates in
    processing of an \openshmem\ \FUNC{shmem\_alltoall} routine. Be careful to
    avoid these situations: If the \VAR{pSync} array is initialized at run time,
    some type of synchronization is needed to ensure that all \acp{PE} in the
    active set have initialized \VAR{pSync} before any of them enter an
    \openshmem\ routine called with the \VAR{pSync} synchronization array.  A
    \VAR{pSync} array may be reused on a subsequent \openshmem\
    \FUNC{shmem\_alltoall} routine only if none of the \acp{PE} in the
    active set are still processing a prior \openshmem\ \FUNC{shmem\_alltoall}
    routine call that used the same \VAR{pSync} array.  In general, this can be
    ensured only by doing some type of synchronization.
}

\begin{apiexamples}

\apicexample
    {This example shows a \FUNC{shmem\_alltoall64} on two long elements among all
    \acp{PE}.}
    {./example_code/shmem_alltoall_example.c}
    {}

\end{apiexamples}

\end{apidefinition}



\subsubsection{\textbf{SHMEM\_ALLTOALLS}}\label{subsec:shmem_alltoalls}
\input{content/shmem_alltoalls.tex}





\subsection{Point-To-Point Synchronization Routines}
The following section discusses \openshmem \ac{API}s that provides a mechanism
for synchronization between two \ac{PE}s based on the value of a symmetric data
object. 

Where appropriate compiler support is available, \openshmem provides
type-generic point-to-point synchronization interfaces via \Celev{} generic
selection. Such type-generic routines are supported for the
``point-to-point synchronization types'' identified in
Table~\ref{p2psynctypes}. Implementations may optionally support additional
types.

\begin{table}[h]
  \begin{center}
    \begin{tabular}{|l|l|}
      \hline
      \TYPE     & \TYPENAME \\ \hline
      short     & short     \\ \hline
      int       & int       \\ \hline
      long      & long      \\ \hline
      long long & longlong  \\ \hline
    \end{tabular}
    \caption{Point-to-Point Synchronization Types and Names}
    \label{p2psynctypes}
  \end{center}
\end{table}


\subsubsection{\textbf{SHMEM\_WAIT}}\label{subsec:shmem_wait}
\apisummary{
    Wait for a variable on the local \ac{PE} to change.
}

\begin{apidefinition}

\begin{C11synopsis}
void shmem_wait(TYPE *ivar, TYPE cmp_value);
void shmem_wait_until(TYPE *ivar, int cmp, TYPE cmp_value);
\end{C11synopsis}
where \TYPE{} is one of the point-to-point synchronization types specified by
Table \ref{p2psynctypes}.

\begin{Csynopsis}
void shmem_<TYPENAME>_wait(TYPE *ivar, TYPE cmp_value);
void shmem_<TYPENAME>_wait_until(TYPE *ivar, int cmp, TYPE cmp_value);
\end{Csynopsis}
where \TYPE{} is one of the point-to-point synchronization types and has a
corresponding \TYPENAME{} specified by Table \ref{p2psynctypes}.

\begin{Fsynopsis}
CALL SHMEM_INT4_WAIT(ivar, cmp_value)
CALL SHMEM_INT4_WAIT_UNTIL(ivar, cmp, cmp_value)
CALL SHMEM_INT8_WAIT(ivar, cmp_value)
CALL SHMEM_INT8_WAIT_UNTIL(ivar, cmp, cmp_value)
CALL SHMEM_WAIT(ivar, cmp_value)
CALL SHMEM_WAIT_UNTIL(ivar, cmp, cmp_value)
\end{Fsynopsis}

\begin{apiarguments}

\apiargument{OUT}{ivar}{A remotely accessible integer variable that is being updated
    by another \ac{PE} \newtext{or the same \ac{PE} in the case of multi-threaded
    \openshmem{}}.  If you are using \CorCpp, the type of \VAR{ivar} should
    match that implied in the SYNOPSIS section.} 
\apiargument{IN}{cmp}{The compare operator that compares \VAR{ivar} with
    \VAR{cmp\_value}.  \VAR{cmp} must be of type integer.  If you are using
    \Fortran, it must  be of  default kind.  If you are using \CorCpp, the type of
    \VAR{cmp} should match that implied in the SYNOPSIS section.}        
\apiargument{IN}{cmp\_value}{\VAR{cmp\_value} must be of type integer.  If you are
    using \CorCpp, the type of \VAR{cmp\_value} should match that implied in the
    SYNOPSIS section.  If  you are using \Fortran, cmp\_value must be an integer of
    the same size and kind as \VAR{ivar}.}

\end{apiarguments}

\apidescription{ 
    \FUNC{shmem\_wait} and \FUNC{shmem\_wait\_until} wait for \VAR{ivar} to be
    changed by a remote write or an atomic operation issued by a
    \oldtext{different} \ac{PE}.
    These  routines can be used for point-to-point direct synchronization.  A call
    to \VAR{shmem\_wait} does not return until a \ac{PE} writes  a value,
    not equal to \VAR{cmp\_value}, into \VAR{ivar} on the waiting \ac{PE}.  A call
    to \FUNC{shmem\_wait\_until} does not return until \oldtext{some  other}
    \newtext{a} \ac{PE} changes
    \VAR{ivar} to satisfy the condition implied by \VAR{cmp} and \VAR{cmp\_value}.
    \oldtext{This mechanism is useful when a \ac{PE} needs to tell another \ac{PE} that it
    has completed some action.}  The \FUNC{shmem\_wait}  routines return when
    \VAR{ivar} is no  longer  equal  to \VAR{cmp\_value}. The
    \FUNC{shmem\_wait\_until} routines return when the compare condition is true.
    The compare condition is defined by the \VAR{ivar}  argument  compared with the
    \VAR{cmp\_value} using the comparison operator, \VAR{cmp}. 
}


\apidesctable{
    If you are using \Fortran, \VAR{ivar} must be a specific sized integer type
    according to the routine being called, as follows:
}{Routine}{Data type}

\apitablerow{shmem\_wait, shmem\_wait\_until}{default INTEGER}
\apitablerow{shmem\_int4\_wait, shmem\_int4\_wait\_until}{INTEGER*4}
\apitablerow{shmem\_int8\_wait, shmem\_int8\_wait\_until}{INTEGER*8}

\apidesctable{
    The following \VAR{cmp} values are supported:
}{CMP Value}{Comparison}

\CorCppFor:\\
\apitablerow{SHMEM\_CMP\_EQ }{  Equal}
\apitablerow{SHMEM\_CMP\_NE}{Not equal}
\apitablerow{SHMEM\_CMP\_GT}{Greater than}
\apitablerow{SHMEM\_CMP\_LE}{Less than or equal to}
\apitablerow{SHMEM\_CMP\_LT}{Less than}
\apitablerow{SHMEM\_CMP\_GE}{Greater than or equal to}

\apireturnvalues{
    None.
}

\apinotes{
    None.
}

\apiimpnotes{
    Implementations must ensure that \FUNC{shmem\_wait} and
    \FUNC{shmem\_wait\_until} do not return before the update of the memory
    indicated by \VAR{ivar} is fully complete.  Partial updates to the memory
    must not cause \FUNC{shmem\_wait} or \FUNC{shmem\_wait\_until} to return.
}


\begin{apiexamples}

\apifexample
{ The following call returns when variable \VAR{ivar} is not equal to 100:}
{./example_code/shmem_wait1_example.f90}
{}

\apifexample
{ The following call to \FUNC{SHMEM\_INT8\_WAIT\_UNTIL} is  equivalent to the
call to \FUNC{SHMEM\_INT8\_WAIT} in example 1:}
{./example_code/shmem_wait2_example.f90}
{}

\apicexample
{The following \CorCpp{} call waits until the value in \VAR{ivar} is set to
be less than zero by a transfer from a remote PE:}
{./example_code/shmem_wait3_example.f90}
{}

\apifexample
{The following \Fortran{} example is in the context of a subroutine:}
{./example_code/shmem_wait4_example.f90}
{}

\end{apiexamples}

\end{apidefinition}






\subsection{Memory Ordering Routines}\label{subsec:memory_order}
The following section discusses \openshmem \ac{API}s that provide mechanisms to
ensure ordering and/or delivery of \OPR{Put}, \ac{AMO}, and memory store
routines to symmetric data objects. 

\subsubsection{\textbf{SHMEM\_FENCE}}\label{subsec:shmem_fence}
\input{content/shmem_fence.tex}

\subsubsection{\textbf{SHMEM\_QUIET}}\label{subsec:shmem_quiet}
\apisummary{
    Waits for completion of all outstanding \PUT{}, \ac{AMO}, memory store,
    and nonblocking \PUT{} and \GET{} routines to symmetric data
    objects issued by a \ac{PE}.
}

\begin{apidefinition}

\begin{Csynopsis}
void @\FuncDecl{shmem\_quiet}@(void);
void @\FuncDecl{shmem\_ctx\_quiet}@(shmem_ctx_t ctx);
\end{Csynopsis}

\begin{Fsynopsis}
CALL @\FuncDecl{SHMEM\_QUIET}@
\end{Fsynopsis}

\begin{apiarguments}
    \apiargument{IN}{ctx}{A context handle specifying the context on which to perform the operation.
        When this argument is not provided, the operation is performed on
        the default context.}
\end{apiarguments}

\apidescription{ 
    The \FUNC{shmem\_quiet} routine ensures completion of \PUT{}, strided \PUT{}, \ac{AMO},
    memory store, and nonblocking \PUT{} and \GET{} routines on
    symmetric data objects issued by the calling \ac{PE} on the given context. All \PUT{}, \ac{AMO},
    memory store, and nonblocking \PUT{} and \GET{} routines to
    symmetric data objects are guaranteed to be completed and visible to all
    \acp{PE} when \FUNC{shmem\_quiet} returns. 
    If \VAR{ctx} has the value \CONST{SHMEM\_CTX\_INVALID}, no operation is
    performed.
}


\apireturnvalues{
    None.
}

\apinotes{ 
    \FUNC{shmem\_quiet} is most useful as a way of ensuring completion of
    several \PUT{}, \ac{AMO}, memory store, and nonblocking \PUT{}
    and \GET{} routines to symmetric data objects initiated by the calling
    \ac{PE}.  For example, one might use \FUNC{shmem\_quiet} to await delivery
    of a block of data before issuing another \PUT{} or nonblocking
    \PUT{} routine, which sets a completion flag on another \ac{PE}.
     \FUNC{shmem\_quiet} is not usually needed if
    \FUNC{shmem\_barrier\_all} or \FUNC{shmem\_barrier} are called.  The barrier
    routines wait for the completion of outstanding writes (\PUT{}, \ac{AMO},
    memory stores, and nonblocking \PUT{} and \GET{} routines) to
    symmetric data objects on all \acp{PE}.

    In an \openshmem program with multithreaded \acp{PE}, it is the
    user's responsibility to ensure ordering between operations issued by the threads
    in a \ac{PE} that target symmetric memory (e.g. \PUT{}, \ac{AMO}, memory stores,
    and nonblocking routines) and calls by threads in that \ac{PE} to
    \FUNC{shmem\_quiet}. The \FUNC{shmem\_quiet} routine can enforce memory store ordering only for the
    calling thread. Thus, to ensure ordering for memory stores performed by a thread that is
    not the thread calling \FUNC{shmem\_quiet}, the update must be made visible to the
    calling thread according to the rules of the memory model associated with
    the threading environment.

     A call to \FUNC{shmem\_quiet} by a thread completes the operations posted prior
     to calling \FUNC{shmem\_quiet}. If the user intends to also complete operations
     issued by a thread that is not the thread calling \FUNC{shmem\_quiet}, the
     user must ensure that the operations are performed prior to the call to
     \FUNC{shmem\_quiet}. This may require the use of a synchronization
     operation provided by the threading package. For example, when using POSIX
     Threads, the user may call the \FUNC{pthread\_barrier\_wait} routine to
     ensure that all threads have issued operations before a thread calls
     \FUNC{shmem\_quiet}.

    \FUNC{shmem\_quiet} does not have an effect on the ordering between memory
    accesses issued by the target PE. \FUNC{shmem\_wait\_until},
    \FUNC{shmem\_test}, \FUNC{shmem\_barrier}, \FUNC{shmem\_barrier\_all} routines
    can be called by the target PE to guarantee ordering of its memory accesses.
}

\begin{apiexamples}

\apicexample
    {The following example uses \FUNC{shmem\_quiet} in a \Cstd[11] program: }
    {./example_code/shmem_quiet_example.c}
    {\VAR{Put1} and \VAR{put2} will be completed and visible before \VAR{put3}
    and \VAR{put4}.}
\end{apiexamples}

\end{apidefinition}


\subsubsection{Synchronization and Communication Ordering in \openshmem}
When using the \openshmem \ac{API}, synchronization, ordering, and completion of
communication become critical. The updates via \PUT{} routines, \acp{AMO} and
store routines on symmetric data cannot be guaranteed until some form of
synchronization or ordering is introduced by the program user. The table below
gives the different synchronization and ordering choices, and the situations
where they may be useful.\\

\begin{tabular}{p{0.2\textwidth} | p{0.7\textwidth}}
\hline 
\textbf{\openshmem  \ac{API}} & \centering \textbf{Working of \openshmem \ac{API}} \tabularnewline
\hline 
\hline 
{Point-to-point synchronization}\\
\FUNC{shmem\_wait}, \FUNC{shmem\_wait\_until} 
&
\raisebox{-\totalheight}{\includegraphics[width=0.7\textwidth]{figures/wait}}
\end{tabular}

\begin{tabular}{p{0.2\textwidth} | p{0.7\textwidth}}
{}
&
{ Waits for a symmetric variable to be updated by a remote \ac{PE}. Should be
used when computation on the local \ac{PE} cannot proceed without the value that
the remote \ac{PE} is to update.} \tabularnewline
\hline 
\end{tabular}

\begin{tabular}{p{0.2\textwidth} | p{0.7\textwidth}}

{Ordering puts issued by a local \ac{PE}} \\
\FUNC{shmem\_fence} 
& 
\raisebox{-\totalheight}{\includegraphics[width=0.7\textwidth]{figures/fence}}
\end{tabular}

\begin{tabular}{p{0.2\textwidth} | p{0.7\textwidth}}
{}
&
All \PUT{} routines, \acp{AMO} and store routines on symmetric data issued to
same \ac{PE}  are guaranteed to be delivered  before Puts (to the same \ac{PE})
issued after the \FUNC{fence} call. \tabularnewline
\hline 
\end{tabular}

\begin{tabular}{p{0.2\textwidth} | p{0.7\textwidth}}
\hline 
\textbf{\openshmem  \ac{API}} & \centering \textbf{Working of \openshmem \ac{API}} \tabularnewline
\hline 
\hline
{Ordering puts issued by all \ac{PE} }\\
\FUNC{shmem\_quiet}
& 
\raisebox{-\totalheight}{\includegraphics[width=0.7\textwidth]{figures/quiet}} 
\end{tabular}

\begin{tabular}{p{0.2\textwidth} | p{0.7\textwidth}}
{}
&
{All \PUT{} routines, \acp{AMO} and store routines on symmetric data issued by a
local \ac{PE} to all  remote \acp{PE} are guaranteed to be completed and visible
once quiet returns. This routine should be used when all remote writes issued by
a local \ac{PE} need to be visible  to all other \acp{PE} before the local
\ac{PE} proceeds. } \tabularnewline
\hline 
\end{tabular}


\begin{tabular}{p{0.2\textwidth} | p{0.7\textwidth}}
Collective synchronization over an \activeset \\
\FUNC{shmem\_barrier}
&  
\raisebox{-\totalheight}{\includegraphics[width=0.7\textwidth]{figures/barrier}} 
\end{tabular}

\begin{tabular}{p{0.2\textwidth} | p{0.7\textwidth}}
{}
&
{All local and remote memory operations issued by all \acp{PE} within the
\activeset{} are guaranteed to be completed before any \ac{PE} in the
\activeset{} returns from the call. Additionally, no \ac{PE} my return from the
barrier until all \acp{PE} in the \activeset{} have entered the same barrier
call. This routine should be used when synchronization as well as completion of
all stores and remote memory updates via \openshmem is required over a sub set
of the executing \acp{PE}.} \tabularnewline
\hline 
\end{tabular}

\begin{tabular}{p{0.2\textwidth} | p{0.7\textwidth}}
\hline 
\textbf{\openshmem  \ac{API}} & \centering \textbf{Working of \openshmem \ac{API}} \tabularnewline
\hline 
\hline
{Collective synchronization over all \acp{PE}} \\
 \FUNC{shmem\_barrier\_all}
& 
\raisebox{-\totalheight}{\includegraphics[width=0.7\textwidth]{figures/barrierall}}
\end{tabular}

\begin{tabular}{p{0.2\textwidth} | p{0.7\textwidth}}
{}
&
{All local and remote memory operations issued by all \acp{PE} are guaranteed to
be completed before any \ac{PE} returns from the call. Additionally no \ac{PE}
shall return from the barrier until all \acp{PE} have entered the same
\FUNC{shmem\_barrier\_all} call. This routine should be used when
synchronization as well as completion of all stores and remote memory updates
via \openshmem is required over all \acp{PE}. } \tabularnewline
\hline 
\end{tabular}
\clearpage







\subsection{Distributed Locking Routines}
The following section discusses \openshmem locks as a mechanism to provide
mutual exclusion. Three routines are available for distributed locking,
\textit{set, test} and \textit{clear}.

\subsubsection{\textbf{SHMEM\_LOCK}}\label{subsec:shmem_lock}
\input{content/shmem_lock.tex}





\subsection{Cache Management}
All of these routines are deprecated and are provided for backwards
compatibility.  Implementations must include all items in this section, and the
routines should function properly and may notify the user about deprecation of
their use.

\subsubsection{\textbf{SHMEM\_CACHE}}\label{subsec:shmem_cache}
\apisummary{
    Controls data cache utilities.
}

\begin{apidefinition}

\begin{DeprecateBlock}
\begin{Csynopsis}
void shmem_clear_cache_inv(void);
void shmem_set_cache_inv(void);
void shmem_clear_cache_line_inv(void *dest);
void shmem_set_cache_line_inv(void *dest);
void shmem_udcflush(void);
void shmem_udcflush_line(void *dest);
\end{Csynopsis}
\end{DeprecateBlock}

\begin{DeprecateBlock}
\begin{Fsynopsis}
CALL SHMEM_CLEAR_CACHE_INV
CALL SHMEM_SET_CACHE_INV
CALL SHMEM_SET_CACHE_LINE_INV(dest)
CALL SHMEM_UDCFLUSH
CALL SHMEM_UDCFLUSH_LINE(dest)
\end{Fsynopsis}
\end{DeprecateBlock}

\begin{apiarguments}

\apiargument{IN}{dest}{A data object that is local to the \ac{PE}.  \VAR{dest}
    can be of any noncharacter type. If you are using \Fortran, it can be of any
    kind.}

\end{apiarguments}

\apidescription{   
    \FUNC{shmem\_set\_cache\_inv} enables automatic cache coherency mode.
    
    \FUNC{shmem\_set\_cache\_line\_inv} enables automatic cache coherency mode for
    the cache line associated with the address of \VAR{dest} only.
    
    \FUNC{shmem\_clear\_cache\_inv} disables automatic cache coherency mode
    previously enabled by \FUNC{shmem\_set\_cache\ \_inv} or
    \FUNC{shmem\_set\_cache\_line\_inv}.
    
    \FUNC{shmem\_udcflush} makes the entire user data cache coherent.
    
    \FUNC{shmem\_udcflush\_line} makes coherent the cache line that corresponds with
    the address specified by \VAR{dest}.
}

\apireturnvalues{
    None.
}

\apinotes{
    These routines have been retained for improved backward compatibility with
    legacy architectures.  They are not required to be supported by implementing
    them as \VAR{no-ops} and where used, they may have no effect on cache line
    states.
}

\begin{apiexamples}

None.

\end{apiexamples}

\end{apidefinition}






\clearpage



\clearpage %%%%%%%%%%%%%%%%%%%%%%%%%%%%%%%%%%%%%%%%%%%%%%%%%%%%%%%%%%%%

\appendix

%defining pagestyle for annex
%\pagestyle{plain} \withlinenumbers
\pagestyle{fancy} \withlinenumbers
\fancyhf{}
\fancyhead[RE, LO]{\leftmark}
\fancyhead[RO, LE]{\thepage}
\fancyfoot[CE, CO]{\thepage}
\renewcommand{\headrulewidth}{0pt}



\chapter{Ordering Rules Interpreted}

The column represents the first operation in the program order and row
represents the second operation in the program order. "Yes" means that the first
and second operation always exectues and completes in order. "F" means that
the second operation executes in order, if there is a shmem\_fence between these 
operations. "Q" means that the second operation completes and visible in
order, if there is a shmem\_quiet between these operations. 

\begin{tabular}{|l|l|l|l|l|l|l|l}
\hline
\textbf{} &\textbf{Put} & \textbf{Get} & \textbf{Put NBI} & \textbf{Get NBI} &
\textbf{Fetch Atomics} & \textbf{Non fetch Atomics} & \textbf{Signal Put}
\tabularnewline\hline
%%
\EnvVarDecl{Put}
    & F/Q
    & Yes
    & Q
    & Q
    & Yes
    & F/Q
    & F/Q
    \tabularnewline\hline
%%
\EnvVarDecl{Get}
    & F/Q
    & Yes
    & Q
    & Q
    & Yes
    & Q
    & Q
    \tabularnewline\hline
%%
\EnvVarDecl{Put NBI}
    & F/Q
    & Yes
    & F/Q
    & Q
    & Yes
    & F/Q
    & F/Q
    \tabularnewline\hline

%%
\EnvVarDecl{Get NBI}
    & F/Q
    & Yes 
    & Q
    & Q
    & Yes
    & Q
    & F/Q
    \tabularnewline\hline

%%
\EnvVarDecl{Fetch Atomics}
    & F/Q
    & Yes
    & Q
    & Q
    & Yes
    & Yes
    & Yes
    \tabularnewline\hline

%%
\EnvVarDecl{Non fetch Atomics}
    & F/Q
    & Yes
    & Q
    & Q
    & Q
    & F/Q
    & Q
    \tabularnewline\hline

\EnvVarDecl{Signal Put}
    & F/Q
    & Yes
    & Q
    & Q
    & Q
    & Q
    & F/Q
    \tabularnewline\hline
    
\end{tabular}

\chapter{Writing OpenSHMEM Programs}
\section*{Incorporating OpenSHMEM into Programs}\label{sec:writing_programs}

The following section describes how to write a ``Hello World" \openshmem program.
To write a ``Hello World" \openshmem program, the user must:

\begin{itemize}
\item Include the header file \HEADER{shmem.h} for \Cstd or \HEADER{shmem.fh} for \Fortran.
\item Add the initialization call \hyperref[subsec:shmem_init]{\FUNC{shmem\_init}}.
\item Use \openshmem calls to query the local \ac{PE} number
    (\hyperref[subsec:shmem_my_pe]{\FUNC{shmem\_my\_pe}}) and the total number
    of \acp{PE} (\hyperref[subsec:shmem_n_pes]{\FUNC{shmem\_n\_pes}}).
\item Add the finalization call \hyperref[subsec:shmem_finalize]{\FUNC{shmem\_finalize}}.
\end{itemize}

In \openshmem, the order in which lines appear in the output is not
deterministic because \acp{PE} execute asynchronously in parallel.

\begin{minipage}{\linewidth}
\vspace{0.1in}
\numberedlisting{caption={``Hello World'' example program in \Cstd},label=openshmem-hello,language=OSH2+C}
                {example_code/hello-openshmem.c}
\outputlisting{language=bash,caption={Possible ordering of expected output with 4 \acp{PE} from the program in Listing~\ref{openshmem-hello}}}
                {example_code/hello-openshmem-c.output}
\vspace{0.1in}
\end{minipage}

\clearpage %%%%%%%%%%%%%%%%%%%%%%%%%%%%%%%%%%%%%%%%%%%%%%%%%%%%%%%%%%%%

\begin{deprecate}
\openshmem also provides a \Fortran API. Listing~\ref{openshmem-hello-f90} shows a similar program written in \Fortran.

\begin{minipage}{\linewidth}
\vspace{0.1in}
\numberedlisting{caption={``Hello World'' example program in \Fortran},label=openshmem-hello-f90,language=OSH2+F}
                {example_code/hello-openshmem.f90}
\outputlisting{language=bash,caption={Possible ordering of expected output with 4 \acp{PE} from the program in Listing~\ref{openshmem-hello-f90}}}
                {example_code/hello-openshmem-f90.output}
\vspace{0.1in}
\end{minipage}
\end{deprecate}

\clearpage %%%%%%%%%%%%%%%%%%%%%%%%%%%%%%%%%%%%%%%%%%%%%%%%%%%%%%%%%%%%

The example in Listing~\ref{openshmem-hello-symmetric} shows a more complex
\openshmem program that illustrates the use of symmetric data objects.
Note the declaration of the \VAR{static short dest} array and its use as the
remote destination in \hyperref[subsec:shmem_put]{\FUNC{shmem\_put}}.

The \KEYWORD{static} keyword makes the \VAR{dest} array symmetric on all \acp{PE}.
Each \ac{PE} is able to transfer data to a remote \dest{} array by simply
specifying to an OpenSHMEM routine such as \hyperref[subsec:shmem_put]{\FUNC{shmem\_put}}
the local address of the symmetric data object that will receive the data.
This local address resolution aids programmability because the address of the
\dest{} need not be exchanged with the active side (\ac{PE} \CONST{0}) prior to
the \acf{RMA} routine.

Conversely, the declaration of the \VAR{short source} array is asymmetric
(local only).
The \source{} object does not need to be symmetric because \PUT{} handles the
references to the \VAR{source} array only on the active (local) side.

\begin{minipage}{\linewidth}
\vspace{0.1in}
\numberedlisting{caption={Example program with symmetric data objects},label=openshmem-hello-symmetric,language=OSH2+C}
                {example_code/writing_shmem_example.c}
\outputlisting{language=bash,caption={Possible ordering of expected output with 4 \acp{PE} from the program in Listing~\ref{openshmem-hello-symmetric}}}
                {example_code/writing_shmem_example.output}
\vspace{0.1in}
\end{minipage}




\chapter{Compiling and Running Programs}\label{sec:compiling}
The \openshmem Specification does not specify how
\openshmem programs are compiled, linked, and run. This section shows some
examples of how wrapper programs are utilized in the \openshmem Reference
Implementation to compile and launch programs.

\section{Compilation}
\subsection*{Programs written in \Cstd}

The \openshmem Reference Implementation provides a wrapper program, named
\textbf{oshcc}, to aid in the compilation of \Cstd programs.
The wrapper may be called as follows:

\begin{lstlisting}[language=bash]
oshcc <compiler options> -o myprogram myprogram.c
\end{lstlisting}
Where the $\langle\mbox{compiler options}\rangle$ are options understood by the
underlying \Cstd compiler called by \textbf{oshcc}.


\subsection*{Programs written in \Cpp}

The \openshmem Reference Implementation provides a wrapper program, named
\textbf{oshc++}, to aid in the compilation of \Cpp programs.
The wrapper may be called as follows:

\begin{lstlisting}[language=bash]
oshc++ <compiler options> -o myprogram myprogram.cpp
\end{lstlisting}
Where the $\langle\mbox{compiler options}\rangle$ are options understood by the
underlying \Cpp compiler called by \textbf{oshc++}.


\subsection*{Programs written in \Fortran}

\begin{deprecate}
The \openshmem Reference Implementation provides a wrapper program, named
\textbf{oshfort}, to aid in the compilation of \Fortran programs.
The wrapper may be called as follows:

\begin{lstlisting}[language=bash]
oshfort <compiler options> -o myprogram myprogram.f
\end{lstlisting}
Where the $\langle\mbox{compiler options}\rangle$ are options understood by the
underlying \Fortran compiler called by \textbf{oshfort}.
\end{deprecate}

\section{Running Programs}

The \openshmem Reference Implementation provides a wrapper program, named
\textbf{oshrun}, to launch \openshmem programs.
The wrapper may be called as follows:

\begin{lstlisting}[language=bash]
oshrun <runner options> -np <#> <program> <program arguments>
\end{lstlisting}
The arguments for \textbf{oshrun} are:

\begin{tabular}{p{0.3\textwidth}p{0.6\textwidth}}
$\langle\mbox{runner options}\rangle$ & {Options passed to the underlying launcher.}\tabularnewline
-np $\langle\mbox{\#}\rangle$ & {The number of \acp{PE} to be used in the execution.}\tabularnewline
$\langle\mbox{program}\rangle$ & {The program executable to be launched.}\tabularnewline
$\langle\mbox{program arguments}\rangle$ & {Flags and other parameters to pass to the program.}\tabularnewline
\end{tabular}




\chapter{Undefined Behavior in OpenSHMEM}\label{sec:undefined}

The \openshmem Specification formalizes the expected behavior of
its library routines.  In cases where routines are improperly used
or the input is not in accordance with the Specification, the behavior
is undefined.

\begin{longtable}{|>{\raggedright}p{0.3\textwidth}|>{\raggedright}p{0.6\textwidth}|}
\hline
\textbf{Inappropriate Usage} & \textbf{Undefined Behavior}\tabularnewline
\hline
\endhead
Uninitialized library & If the \openshmem library is not initialized,
calls to non-initializing \openshmem routines have undefined
behavior.  For example, an implementation may try to continue or may abort
immediately upon an \openshmem call into the uninitialized library.
\tabularnewline
\hline
Multiple calls to initialization routines & In an \openshmem program where
the initialization routines \FUNC{shmem\_init} or \FUNC{shmem\_init\_thread}
have already been called, any subsequent calls to these initialization routines
result in undefined behavior.
\tabularnewline
\hline
Accessing non-existent \acp{PE} & If a communications routine accesses a
non-existent \ac{PE}, then the \openshmem library may handle this
situation in an implementation-defined way.  For example, the library may report
an error message saying that the \ac{PE} accessed is outside the range of
accessible \acp{PE}, or may exit without a warning.\tabularnewline
\hline
Use of non-symmetric variables & Some routines require remotely accessible
variables to perform their function.  For example, a \PUT{} to a non-symmetric variable may
be trapped where possible and the library may abort the program.  Another
implementation may choose to continue execution with or without a warning.
\tabularnewline
\hline
Non-symmetric allocation of symmetric memory & The symmetric memory management routines are
collectives. For example, all \acp{PE} in the program must call
\FUNC{shmem\_malloc} with the same \VAR{size} argument.  Program behavior after a
mismatched \FUNC{shmem\_malloc} call is undefined.\tabularnewline
\hline
Use of null pointers with non-zero \VAR{len} specified & In any \openshmem routine
that takes a pointer and \VAR{len} describing the number of elements in that
pointer, a null pointer may not be given unless the corresponding \VAR{len} is also
specified as zero. Otherwise, the resulting behavior is undefined.
The following cases summarize this behavior:
\begin{itemize}
    \item \VAR{len} is 0, pointer is null: supported.
    \item \VAR{len} is not 0, pointer is null: undefined behavior.
    \item \VAR{len} is 0, pointer is non-null: supported.
    \item \VAR{len} is not 0, pointer is non-null: supported.
\end{itemize}
\tabularnewline
\hline
\end{longtable}




\chapter{History of OpenSHMEM}\label{sec:openshmem_history}

SHMEM has a long history as a parallel-programming model and has been
extensively used on a number of products since 1993, including the Cray T3D,
Cray X1E, Cray XT3 and XT4, \ac{SGI} Origin, \ac{SGI} Altix, Quadrics-based
clusters, and InfiniBand-based clusters.

\begin{itemize}
\item SHMEM Timeline
  \begin{itemize}
  \item Cray SHMEM
    \begin{itemize}
    \item SHMEM first introduced by Cray Research, Inc.\ in 1993 for Cray T3D
    \item Cray was acquired by \ac{SGI} in 1996
    \item Cray was acquired by Tera in 2000 (MTA)
    \item Platforms: Cray T3D, T3E, C90, J90, SV1, SV2, X1, X2, XE, XMT, XT
    \end{itemize}
  \item \ac{SGI} SHMEM
    \begin{itemize}
    \item \ac{SGI} acquired Cray Research, Inc.\ and SHMEM was integrated into
      \ac{SGI}'s Message Passing Toolkit (MPT)
    \item \ac{SGI} currently owns the rights to SHMEM and \openshmem
    \item Platforms: Origin, Altix 4700, Altix XE, ICE, UV
    \item \ac{SGI} was acquired by Rackable Systems in 2009
    \item \ac{SGI} and \ac{OSSS} signed a
      SHMEM trademark licensing agreement in 2010
    \item \ac{HPE} acquired {SGI} in 2016
    \end{itemize}
  \end{itemize}
\end{itemize}

A listing of \openshmem implementations can be found on
\url{http://www.openshmem.org/}.








\chapter{OpenSHMEM Specification and Deprecated API}\label{sec:dep_api}

\section{Overview}\label{subsec:dep_overview}
\TableIndex{Deprecated API}
For the \openshmem Specification, deprecation is the process of identifying
API that is supported but no longer recommended for use by users.
The deprecated API \textbf{must} be supported until clearly
indicated as otherwise by the Specification.
This chapter records the API or functionality that have been deprecated, the
version of the \openshmem Specification that effected the deprecation, and the
most recent version of the \openshmem Specification in which the feature was
supported before removal.

\begin{center}
\scriptsize
\begin{longtable}{|l|c|c|l|}
    \hline
    \textbf{Deprecated API}
    & \textbf{Deprecated Since}
    & \textbf{Last Version Supported}
    & \textbf{Replaced By} \\
    \hline
    \endhead
    Header Directory: \hyperref[subsec:dep_rationale:mpp]{\HEADER{mpp}} & 1.1 & Current & (none) \\ \hline
    \CorCpp: \hyperref[subsec:start_pes]{\FuncRef{start\_pes}} & 1.2 & Current & \hyperref[subsec:shmem_init]{\FUNC{shmem\_init}} \\ \hline
    \Fortran: \hyperref[subsec:start_pes]{\FuncRef{START\_PES}} & 1.2 & Current & \hyperref[subsec:shmem_init]{\FUNC{SHMEM\_INIT}} \\ \hline
    \hyperref[subsec:start_pes]{Implicit finalization} & 1.2 & Current & \hyperref[subsec:shmem_finalize]{\FUNC{shmem\_finalize}} \\ \hline
    \CorCpp: \FuncRef{\_my\_pe} & 1.2 & Current & \hyperref[subsec:shmem_my_pe]{\FUNC{shmem\_my\_pe}} \\ \hline
    \CorCpp: \FuncRef{\_num\_pes} & 1.2 & Current & \hyperref[subsec:shmem_n_pes]{\FUNC{shmem\_n\_pes}} \\ \hline
    \Fortran: \FuncRef{MY\_PE} & 1.2 & Current & \hyperref[subsec:shmem_my_pe]{\FUNC{SHMEM\_MY\_PE}} \\ \hline
    \Fortran: \FuncRef{NUM\_PES} & 1.2 & Current & \hyperref[subsec:shmem_n_pes]{\FUNC{SHMEM\_N\_PES}} \\ \hline
    \CorCpp: \FuncRef{shmalloc} & 1.2 & Current & \hyperref[subsec:shfree]{\FUNC{shmem\_malloc}} \\ \hline
    \CorCpp: \FuncRef{shfree} & 1.2 & Current & \hyperref[subsec:shfree]{\FUNC{shmem\_free}} \\ \hline
    \CorCpp: \FuncRef{shrealloc} & 1.2 & Current & \hyperref[subsec:shfree]{\FUNC{shmem\_realloc}} \\ \hline
    \CorCpp: \FuncRef{shmemalign} & 1.2 & Current & \hyperref[subsec:shfree]{\FUNC{shmem\_align}} \\ \hline
    \Fortran: \FuncRef{SHMEM\_PUT} & 1.2 & Current & \hyperref[subsec:shmem_put]{\FUNC{SHMEM\_PUT8} or \FUNC{SHMEM\_PUT64}} \\ \hline
    \minitab{\CorCpp: \hyperref[subsec:shmem_cache]{\FuncRef{shmem\_clear\_cache\_inv}}
        \\ \Fortran: \hyperref[subsec:shmem_cache]{\FuncRef{SHMEM\_CLEAR\_CACHE\_INV}}}
        & 1.3 & Current & (none) \\ \hline
    \CorCpp: \hyperref[subsec:shmem_cache]{\FuncRef{shmem\_clear\_cache\_line\_inv}} & 1.3 & Current & (none) \\ \hline
    \minitab{\CorCpp: \hyperref[subsec:shmem_cache]{\FuncRef{shmem\_set\_cache\_inv}}
        \\ \Fortran: \hyperref[subsec:shmem_cache]{\FuncRef{SHMEM\_SET\_CACHE\_INV}}}
        & 1.3 & Current & (none) \\ \hline
    \minitab{\CorCpp: \hyperref[subsec:shmem_cache]{\FuncRef{shmem\_set\_cache\_line\_inv}}
        \\ \Fortran: \hyperref[subsec:shmem_cache]{\FuncRef{SHMEM\_SET\_CACHE\_LINE\_INV}}}
        & 1.3 & Current & (none) \\ \hline
    \minitab{\CorCpp: \hyperref[subsec:shmem_cache]{\FuncRef{shmem\_udcflush}}
        \\ \Fortran: \hyperref[subsec:shmem_cache]{\FuncRef{SHMEM\_UDCFLUSH}}}
        & 1.3 & Current & (none) \\ \hline
    \minitab{\CorCpp: \hyperref[subsec:shmem_cache]{\FuncRef{shmem\_udcflush\_line}}
        \\ \Fortran: \hyperref[subsec:shmem_cache]{\FuncRef{SHMEM\_UDCFLUSH\_LINE}}}
        & 1.3 & Current & (none) \\ \hline
    \LibConstRef{\_SHMEM\_SYNC\_VALUE}         & 1.3 & Current & \hyperref[subsec:library_constants]{\CONST{SHMEM\_SYNC\_VALUE}} \\ \hline
    \LibConstRef{\_SHMEM\_BARRIER\_SYNC\_SIZE} & 1.3 & Current & \hyperref[subsec:library_constants]{\CONST{SHMEM\_BARRIER\_SYNC\_SIZE}} \\ \hline
    \LibConstRef{\_SHMEM\_BCAST\_SYNC\_SIZE}   & 1.3 & Current & \hyperref[subsec:library_constants]{\CONST{SHMEM\_BCAST\_SYNC\_SIZE}} \\ \hline
    \LibConstRef{\_SHMEM\_COLLECT\_SYNC\_SIZE} & 1.3 & Current & \hyperref[subsec:library_constants]{\CONST{SHMEM\_COLLECT\_SYNC\_SIZE}} \\ \hline
    \LibConstRef{\_SHMEM\_REDUCE\_SYNC\_SIZE}  & 1.3 & Current & \hyperref[subsec:library_constants]{\CONST{SHMEM\_REDUCE\_SYNC\_SIZE}} \\ \hline
    \LibConstRef{\_SHMEM\_REDUCE\_MIN\_WRKDATA\_SIZE} & 1.3 & Current & \hyperref[subsec:library_constants]{\CONST{SHMEM\_REDUCE\_MIN\_WRKDATA\_SIZE}} \\ \hline
    \LibConstRef{\_SHMEM\_MAJOR\_VERSION} & 1.3 & Current & \hyperref[subsec:library_constants]{\CONST{SHMEM\_MAJOR\_VERSION}} \\ \hline
    \LibConstRef{\_SHMEM\_MINOR\_VERSION} & 1.3 & Current & \hyperref[subsec:library_constants]{\CONST{SHMEM\_MINOR\_VERSION}} \\ \hline
    \LibConstRef{\_SHMEM\_MAX\_NAME\_LEN} & 1.3 & Current & \hyperref[subsec:library_constants]{\CONST{SHMEM\_MAX\_NAME\_LEN}} \\ \hline
    \LibConstRef{\_SHMEM\_VENDOR\_STRING} & 1.3 & Current & \hyperref[subsec:library_constants]{\CONST{SHMEM\_VENDOR\_STRING}} \\ \hline
    \LibConstRef{\_SHMEM\_CMP\_EQ} & 1.3 & Current & \hyperref[subsec:library_constants]{\CONST{SHMEM\_CMP\_EQ}} \\ \hline
    \LibConstRef{\_SHMEM\_CMP\_NE} & 1.3 & Current & \hyperref[subsec:library_constants]{\CONST{SHMEM\_CMP\_NE}} \\ \hline
    \LibConstRef{\_SHMEM\_CMP\_LT} & 1.3 & Current & \hyperref[subsec:library_constants]{\CONST{SHMEM\_CMP\_LT}} \\ \hline
    \LibConstRef{\_SHMEM\_CMP\_LE} & 1.3 & Current & \hyperref[subsec:library_constants]{\CONST{SHMEM\_CMP\_LE}} \\ \hline
    \LibConstRef{\_SHMEM\_CMP\_GT} & 1.3 & Current & \hyperref[subsec:library_constants]{\CONST{SHMEM\_CMP\_GT}} \\ \hline
    \LibConstRef{\_SHMEM\_CMP\_GE} & 1.3 & Current & \hyperref[subsec:library_constants]{\CONST{SHMEM\_CMP\_GE}} \\ \hline
    \EnvVarRef{SMA\_VERSION}         & 1.4 & Current & \hyperref[subsec:environment_variables]{\ENVVAR{SHMEM\_VERSION}} \\ \hline
    \EnvVarRef{SMA\_INFO}            & 1.4 & Current & \hyperref[subsec:environment_variables]{\ENVVAR{SHMEM\_INFO}} \\ \hline
    \EnvVarRef{SMA\_SYMMETRIC\_SIZE} & 1.4 & Current & \hyperref[subsec:environment_variables]{\ENVVAR{SHMEM\_SYMMETRIC\_SIZE}} \\ \hline
    \EnvVarRef{SMA\_DEBUG}           & 1.4 & Current & \hyperref[subsec:environment_variables]{\ENVVAR{SHMEM\_DEBUG}} \\ \hline
    \minitab{\CorCpp: \FuncRef{shmem\_wait}
        \\ \CorCpp: \FuncRef{shmem\_\FuncParam{TYPENAME}\_wait}}
        & 1.4 & Current & See \textbf{Notes} for \hyperref[subsec:shmem_wait_until]{\FUNC{shmem\_wait\_until}} \\ \hline
    \CorCpp: \FuncRef{shmem\_wait\_until} & 1.4 & Current
        & \Cstd[11]: \hyperref[subsec:shmem_wait_until]{\FUNC{shmem\_wait\_until}}, \CorCpp: \hyperref[subsec:shmem_wait_until]{\FUNC{shmem\_long\_wait\_until}} \\ \hline
    \minitab{\Cstd[11]: \FuncRef{shmem\_fetch}
        \\ \CorCpp: \FuncRef{shmem\_\FuncParam{TYPENAME}\_fetch}}
        & 1.4 & Current & \hyperref[subsec:shmem_atomic_fetch]{\FUNC{shmem\_atomic\_fetch}} \\ \hline
    \minitab{\Cstd[11]: \FuncRef{shmem\_set}
        \\ \CorCpp: \FuncRef{shmem\_\FuncParam{TYPENAME}\_set}}
        & 1.4 & Current & \hyperref[subsec:shmem_atomic_set]{\FUNC{shmem\_atomic\_set}} \\ \hline
    \minitab{\Cstd[11]: \FuncRef{shmem\_cswap}
        \\ \CorCpp: \FuncRef{shmem\_\FuncParam{TYPENAME}\_cswap}}
        & 1.4 & Current & \hyperref[subsec:shmem_atomic_compare_swap]{\FUNC{shmem\_atomic\_compare\_swap}} \\ \hline
    \minitab{\Cstd[11]: \FuncRef{shmem\_swap}
        \\ \CorCpp: \FuncRef{shmem\_\FuncParam{TYPENAME}\_swap}}
        & 1.4 & Current & \hyperref[subsec:shmem_atomic_swap]{\FUNC{shmem\_atomic\_swap}} \\ \hline
    \minitab{\Cstd[11]: \FuncRef{shmem\_finc}
        \\ \CorCpp: \FuncRef{shmem\_\FuncParam{TYPENAME}\_finc}}
        & 1.4 & Current & \hyperref[subsec:shmem_atomic_fetch_inc]{\FUNC{shmem\_atomic\_fetch\_inc}} \\ \hline
    \minitab{\Cstd[11]: \FuncRef{shmem\_inc}
        \\ \CorCpp: \FuncRef{shmem\_\FuncParam{TYPENAME}\_inc}}
        & 1.4 & Current & \hyperref[subsec:shmem_atomic_inc]{\FUNC{shmem\_atomic\_inc}} \\ \hline
    \minitab{\Cstd[11]: \FuncRef{shmem\_fadd}
        \\ \CorCpp: \FuncRef{shmem\_\FuncParam{TYPENAME}\_fadd}}
        & 1.4 & Current & \hyperref[subsec:shmem_atomic_fetch_add]{\FUNC{shmem\_atomic\_fetch\_add}} \\ \hline
    \minitab{\Cstd[11]: \FuncRef{shmem\_add}
        \\ \CorCpp: \FuncRef{shmem\_\FuncParam{TYPENAME}\_add}}
        & 1.4 & Current & \hyperref[subsec:shmem_atomic_add]{\FUNC{shmem\_atomic\_add}} \\ \hline
    Entire \Fortran API & 1.4 & Current & (none) \\ \hline
    \end{longtable}
\end{center}

\section{Deprecation Rationale}\label{subsec:dep_rationale}

\subsection{Header Directory: \HEADER{mpp}}
\label{subsec:dep_rationale:mpp}
In addition to the default system header paths, \openshmem implementations
must provide all \openshmem-specified header files from the \HEADER{mpp}
header directory such that these headers can be referenced in \CorCpp as
\begin{lstlisting}[language=]
#include <mpp/shmem.h>
#include <mpp/shmemx.h>
\end{lstlisting}
and in \Fortran as
\begin{lstlisting}[language=]
include 'mpp/shmem.fh'
include 'mpp/shmemx.fh'
\end{lstlisting}
for backwards compatibility with \ac{SGI} SHMEM.

\subsection{\CorCpp: \FUNC{start\_pes}}
The \CorCpp routine \FUNC{start\_pes} includes an unnecessary initialization
argument that is remnant of historical \emph{SHMEM} implementations and no
longer reflects the requirements of modern \openshmem implementations.
Furthermore, the naming of \FUNC{start\_pes} does not include the standardized
\shmemprefixLC{} naming prefix. This routine has been deprecated and
\openshmem users are encouraged to use \FUNC{shmem\_init} instead.

\subsection{Implicit Finalization}
Implicit finalization was deprecated and replaced with explicit finalization using the
\FUNC{shmem\_finalize} routine.  Explicit finalization improves portability and
also improves interoperability with profiling and debugging tools.

\subsection{\CorCpp: \FUNC{\_my\_pe}, \FUNC{\_num\_pes}, \FUNC{shmalloc},
    \FUNC{shfree}, \FUNC{shrealloc}, \FUNC{shmemalign}}
The \CorCpp routines \FUNC{\_my\_pe}, \FUNC{\_num\_pes}, \FUNC{shmalloc},
\FUNC{shfree}, \FUNC{shrealloc}, and \FUNC{shmemalign} were deprecated in order
to normalize the \openshmem \ac{API} to use \shmemprefixLC{} as the standard
prefix for all routines.

\subsection{\textit{Fortran}: \FUNC{START\_PES}, \FUNC{MY\_PE}, \FUNC{NUM\_PES}} %% WARNING: Issue #66.
The \Fortran routines \FUNC{START\_PES}, \FUNC{MY\_PE}, and \FUNC{NUM\_PES}
were deprecated in order to minimize the API differences from the deprecation
of \CorCpp routines \FUNC{start\_pes}, \FUNC{\_my\_pe}, and \FUNC{\_num\_pes}.

\subsection{\textit{Fortran}: \FUNC{SHMEM\_PUT}} %% WARNING: Issue #66.
The \Fortran routine \FUNC{SHMEM\_PUT} is defined only for the \Fortran
\ac{API} and is semantically identical to \Fortran routines
\FUNC{SHMEM\_PUT8} and \FUNC{SHMEM\_PUT64}.  Since \FUNC{SHMEM\_PUT8} and
\FUNC{SHMEM\_PUT64} have defined equivalents in the \CorCpp interface,
\FUNC{SHMEM\_PUT} is ambiguous and has been deprecated.

\subsection{SHMEM\_CACHE}
The \FUNC{SHMEM\_CACHE} \ac{API}
\begin{center}
\begin{tabular}{ll}
    \CorCpp: & \Fortran: \\
    \FUNC{shmem\_clear\_cache\_inv}     & \FUNC{SHMEM\_CLEAR\_CACHE\_INV} \\
    \FUNC{shmem\_set\_cache\_inv}       & \FUNC{SHMEM\_SET\_CACHE\_INV} \\
    \FUNC{shmem\_set\_cache\_line\_inv} & \FUNC{SHMEM\_SET\_CACHE\_LINE\_INV} \\
    \FUNC{shmem\_udcflush}              & \FUNC{SHMEM\_UDCFLUSH} \\
    \FUNC{shmem\_udcflush\_line}        & \FUNC{SHMEM\_UDCFLUSH\_LINE} \\
    \FUNC{shmem\_clear\_cache\_line\_inv} \\
\end{tabular}
\end{center}
was originally implemented for systems with cache-management instructions.
This API has largely gone unused on cache-coherent system architectures.
\FUNC{SHMEM\_CACHE} has been deprecated.

\subsection{\CONST{\_SHMEM\_*} Library Constants}
The library constants
\begin{center}
\begin{tabular}{ll}
    \CONST{\_SHMEM\_SYNC\_VALUE}         & \CONST{\_SHMEM\_MAX\_NAME\_LEN} \\
    \CONST{\_SHMEM\_BARRIER\_SYNC\_SIZE} & \CONST{\_SHMEM\_VENDOR\_STRING} \\
    \CONST{\_SHMEM\_BCAST\_SYNC\_SIZE}   & \CONST{\_SHMEM\_CMP\_EQ} \\
    \CONST{\_SHMEM\_COLLECT\_SYNC\_SIZE} & \CONST{\_SHMEM\_CMP\_NE} \\
    \CONST{\_SHMEM\_REDUCE\_SYNC\_SIZE}  & \CONST{\_SHMEM\_CMP\_LT} \\
    \CONST{\_SHMEM\_REDUCE\_MIN\_WRKDATA\_SIZE} & \CONST{\_SHMEM\_CMP\_LE} \\
    \CONST{\_SHMEM\_MAJOR\_VERSION}      & \CONST{\_SHMEM\_CMP\_GT} \\
    \CONST{\_SHMEM\_MINOR\_VERSION}      & \CONST{\_SHMEM\_CMP\_GE} \\
\end{tabular}
\end{center}
do not adhere to the \Cstd standard's reserved identifiers and the \Cpp
standard's reserved names.  These constants were deprecated and replaced
with corresponding constants of prefix \shmemprefix{} that adhere to \CorCpp{}
and \Fortran naming conventions.

\subsection{\ENVVAR{SMA\_*} Environment Variables}\label{subsec:deprecate-sma-env}
The environment variables \ENVVAR{SMA\_VERSION}, \ENVVAR{SMA\_INFO},
\ENVVAR{SMA\_SYMMETRIC\_SIZE}, and \ENVVAR{SMA\_DEBUG}
were deprecated in order to normalize the \openshmem \ac{API} to use
\shmemprefix{} as the standard prefix for all environment variables.

\subsection{\CorCpp: \FUNC{shmem\_wait}}
The \CorCpp interface for \FUNC{shmem\_wait} and \FUNC{shmem\_\FuncParam{TYPENAME}\_wait}
was identified as unintuitive with respect to
the comparison operation it performed.  As \FUNC{shmem\_wait} can be trivially
replaced by \FUNC{shmem\_wait\_until} where \VAR{cmp} is
\CONST{SHMEM\_CMP\_NE}, the \FUNC{shmem\_wait} interface was deprecated in
favor of \FUNC{shmem\_wait\_until}, which makes the comparison operation
explicit and better communicates the developer's intent.

\subsection{\CorCpp: \FUNC{shmem\_wait\_until}}
The \CTYPE{long}-typed \CorCpp routine \FUNC{shmem\_wait\_until} was deprecated
in favor of the \Cstd[11] type-generic interface of the same name or the
explicitly typed \CorCpp routine \FUNC{shmem\_long\_wait\_until}.

\subsection{\textit{C11} and \CorCpp: \FUNC{shmem\_fetch}, \FUNC{shmem\_set}, %% Issue #66.
    \FUNC{shmem\_cswap}, \FUNC{shmem\_swap}, \FUNC{shmem\_finc},
    \FUNC{shmem\_inc}, \FUNC{shmem\_fadd}, \FUNC{shmem\_add}}
The \Cstd[11] and \CorCpp interfaces for
\begin{center}
\begin{tabular}{ll}
    \Cstd[11]: & \CorCpp: \\
    \FUNC{shmem\_fetch} & \FUNC{shmem\_\FuncParam{TYPENAME}\_fetch} \\
    \FUNC{shmem\_set}   & \FUNC{shmem\_\FuncParam{TYPENAME}\_set}   \\
    \FUNC{shmem\_cswap} & \FUNC{shmem\_\FuncParam{TYPENAME}\_cswap} \\
    \FUNC{shmem\_swap}  & \FUNC{shmem\_\FuncParam{TYPENAME}\_swap}  \\
    \FUNC{shmem\_finc}  & \FUNC{shmem\_\FuncParam{TYPENAME}\_finc}  \\
    \FUNC{shmem\_inc}   & \FUNC{shmem\_\FuncParam{TYPENAME}\_inc}   \\
    \FUNC{shmem\_fadd}  & \FUNC{shmem\_\FuncParam{TYPENAME}\_fadd}  \\
    \FUNC{shmem\_add}   & \FUNC{shmem\_\FuncParam{TYPENAME}\_add}   \\
\end{tabular}
\end{center}
were deprecated and replaced with
similarly named interfaces within the \FUNC{shmem\_atomic\_*} namespace
in order to more clearly identify these calls as performing atomic operations.
In addition, the abbreviated names ``cswap'', ``finc'', and ``fadd'' were
expanded for clarity to ``compare\_swap'', ``fetch\_inc'', and ``fetch\_add''.

\subsection{\textit{Fortran} API}\label{subsec:deprecate-fortran} %% WARNING: Issue #66.
The entire \openshmem \Fortran API was deprecated because of a general lack of
use and a lack of conformance with legacy \Fortran standards. In lieu of an
extensive update of the \Fortran API, \Fortran users are encouraged to
leverage the \openshmem Specification's \Cstd API through the
\Fortran--\Cstd interoperability initially standardized by \Fortran[2003]%
\footnote{Formally, \Fortran[2003] is known as ISO/IEC~1539-1:2004(E).}.





\chapter{Changes to this Document}\label{sec:changelog}

\section{Version 1.5}
Major changes in \openshmem[1.5] include \dots

The following list describes the specific changes in \openshmem[1.5]:
\begin{itemize}
%
\item Added a new multiple-element point-to-point synchronization API with
  the functions: \FUNC{shmem\_wait\_until\_all}, \FUNC{shmem\_wait\_until\_any},
  \FUNC{shmem\_wait\_until\_some}, \FUNC{shmem\_test\_all},
  \FUNC{shmem\_test\_any}, and \FUNC{shmem\_test\_some}.
  \\See Sections \ref{subsec:shmem_wait_until_all},
  \ref{subsec:shmem_wait_until_any}, \ref{subsec:shmem_wait_until_some},
  \ref{subsec:shmem_test_all}, \ref{subsec:shmem_test_any}, and
  \ref{subsec:shmem_test_some}.
%
\item Added \openshmem profiling interface.
  \\ See Section~\ref{sec:openshmem_profiling_interface}.
%
\item Specified the validity of communication contexts, added the constant
  \CONST{SHMEM\_CTX\_INVALID}, and clarified the behavior of
  \FUNC{shmem\_ctx\_*} routines on invalid contexts.
  \\ See Section~\ref{sec:ctx}.
%
\item Clarified \ac{PE} active set requirements.
    \\See Section~\ref{subsec:coll}.
%
\item Clarified that when the \VAR{size} argument is zero, symmetric heap
    allocation routines perform no action and return a null pointer; that
    symmetric heap management routines that perform no action do not perform a
    barrier; and that the \VAR{alignment} argument to \FUNC{shmem\_align} must
    be power of two multiple of \CONST{sizeof(void*)}.
    \\See Section~\ref{subsec:shfree}.
%
\item Clarified that the \openshmem lock API provides a non-reentrant mutex and
    that \FUNC{shmem\_clear\_lock} performs a quiet operation on the default
    context.
    \\See Section~\ref{subsec:shmem_lock}
%
\item Clarified the atomicity guarantees of the \openshmem memory model.
    \\See Section~\ref{subsec:amo_guarantees}.
%
\end{itemize}

\section{Version 1.4}
Major changes in \openshmem[1.4] include
multithreading support,
\emph{contexts} for communication management,
\FUNC{shmem\_sync},
\FUNC{shmem\_calloc},
expanded type support,
a new namespace for atomic operations,
atomic bitwise operations,
\FUNC{shmem\_test} for nonblocking point-to-point synchronization,
and \Cstd[11] type-generic interfaces for point-to-point synchronization.

The following list describes the specific changes in \openshmem[1.4]:
\begin{itemize}
%
\item New communication management API, including \FUNC{shmem\_ctx\_create};
    \FUNC{shmem\_ctx\_destroy}; and additional RMA, AMO, and memory ordering
    routines that accept \CTYPE{shmem\_ctx\_t} arguments.
\\See Section \ref{sec:ctx}.
%
\item New API \FUNC{shmem\_sync\_all} and \FUNC{shmem\_sync} to provide \ac{PE}
    synchronization without completing pending communication operations.
    \\See Sections \ref{subsec:shmem_sync_all} and \ref{subsec:shmem_sync}.
%
\item Clarified that the \openshmem extensions header files are required, even when empty.
\\See Section~\ref{subsec:bindings}.
%
\item Clarified that the \FUNC{SHMEM\_GET64} and \FUNC{SHMEM\_GET64\_NBI}
    routines are included in the \Fortran language bindings.\\
    See Sections \ref{subsec:shmem_get} and \ref{subsec:shmem_get_nbi}.
%
\item Clarified that \FUNC{shmem\_init} must be matched with a call to
    \FUNC{shmem\_finalize}.
\\See Sections \ref{subsec:shmem_init} and \ref{subsec:shmem_finalize}.
%
\item Added the \CONST{SHMEM\_SYNC\_SIZE} constant.
\\See Section \ref{subsec:library_constants}.
%
\item Added type-generic interfaces for \FUNC{shmem\_wait\_until}.
\\ See Section \ref{subsec:shmem_wait_until}.
%
\item Removed the \VAR{volatile} qualifiers from the \VAR{ivar} arguments to
\FUNC{shmem\_wait} routines and the \VAR{lock} arguments in the lock API.
\emph{Rationale: Volatile qualifiers were added to several API routines in
\openshmem[1.3]; however, they were later found to be unnecessary.}
\\ See Sections \ref{subsec:shmem_wait_until} and \ref{subsec:shmem_lock}.
%
\item Deprecated the \VAR{SMA\_}* environment variables and added equivalent
\VAR{SHMEM\_}* environment variables.
\\ See Section \ref{subsec:environment_variables}.
%
\item Added the \Cstd[11] \CTYPE{\_Noreturn} function specifier to
\FUNC{shmem\_global\_exit}.
\\ See Section \ref{subsec:shmem_global_exit}.
%
\item Clarified ordering semantics of memory ordering, point-to-point synchronization, and collective
synchronization routines.
%
\item Clarified deprecation overview and added deprecation rationale in Annex F.
\\See Section \ref{sec:dep_api}.
%
\item Deprecated header directory \HEADER{mpp}.
\\See Section \ref{sec:dep_api}.
%
\item Deprecated the \FUNC{shmem\_wait} functions and the \CTYPE{long}-typed \CorCpp \FUNC{shmem\_wait\_until} function.
\\ See Section \ref{subsec:p2p_intro}.
%
\item Added the \FUNC{shmem\_test} functions.
\\ See Section \ref{subsec:p2p_intro}.
%
\item Added the \FUNC{shmem\_calloc} function.
\\ See Section \ref{subsec:shmem_calloc}.
%
\item Introduced the thread safe semantics that define the interaction between
    \openshmem routines and user threads.
\\See Section \ref{subsec:thread_support}.
%
\item Added the new routine \FUNC{shmem\_init\_thread} to initialize the
    \openshmem library with one of the defined thread levels.
\\See Section \ref{subsec:shmem_init_thread}.
%
\item Added the new routine \FUNC{shmem\_query\_thread} to query the thread
    level provided by the \openshmem implementation.
\\See Section \ref{subsec:shmem_query_thread}.
%
\item Clarified the semantics of \FUNC{shmem\_quiet} for a multithreaded
    \openshmem \ac{PE}.
\\See Section \ref{subsec:shmem_quiet}
%
\item Revised the description of \FUNC{shmem\_barrier\_all} for a multithreaded
    \openshmem \ac{PE}.
\\See Section \ref{subsec:shmem_barrier_all}
%
\item Revised the description of \FUNC{shmem\_wait} for a multithreaded
    \openshmem \ac{PE}.
\\See Section \ref{subsec:shmem_wait_until}
%
\item Clarified description for \CONST{SHMEM\_VENDOR\_STRING}.
\\See Section \ref{subsec:library_constants}.
%
\item Clarified description for \CONST{SHMEM\_MAX\_NAME\_LEN}.
\\See Section \ref{subsec:library_constants}.
%
\item Clarified API description for \FUNC{shmem\_info\_get\_name}.
\\See Section \ref{subsec:shmem_info_get_name}.
%
\item Expanded the type support for RMA, AMO, and point-to-point
    synchronization operations.
\\ See Tables \ref{stdrmatypes}, \ref{stdamotypes}, \ref{extamotypes}, and
    \ref{p2psynctypes}
%
\item Renamed AMO operations to use \FUNC{shmem\_atomic\_*} prefix and
      deprecated old AMO routines.
\\ See Section \ref{sec:amo}.
%
\item Added fetching and non-fetching bitwise AND, OR, and XOR atomic
      operations.
\\ See Section \ref{sec:amo}.
%
\item Deprecated the entire \Fortran API.
%
\item Replaced the \CTYPE{complex} macro in complex-typed reductions with the
      \Cstd[99] (and later) type specifier \CTYPE{\_Complex} to remove an
      implicit dependence on \HEADER{complex.h}.
\\ See Section \ref{subsec:shmem_reductions}.
%
\item Clarified that complex-typed reductions in C are optionally supported.
\\ See Section \ref{subsec:shmem_reductions}.
%
\end{itemize}




\section{Version 1.3}
Major changes in \openshmem[1.3] include the addition of
nonblocking \ac{RMA} operations,
atomic \PUT{} and \GET{} operations,
all-to-all collectives,
and \Cstd[11] type-generic interfaces for \ac{RMA} and \ac{AMO} operations.

The following list describes the specific changes in \openshmem[1.3]:
\begin{itemize}
%
\item Clarified implementation of \acp{PE} as threads.
%
\item Added \CTYPE{const} to every read-only pointer argument.
%
\item Clarified definition of \OPR{Fence}.
\\See Section \ref{subsec:programming_model}.
%
\item Clarified implementation of symmetric memory allocation.
\\See Section \ref{subsec:memory_model}.
%
\item Restricted atomic operation guarantees to other atomic operations with the same datatype.
\\See Section \ref{subsec:amo_guarantees}.
%
\item Deprecation of all constants that start with \CONST{\_SHMEM\_*}.
\\See Section \ref{subsec:library_constants}.
%
\item Added a type-generic interface to \openshmem \ac{RMA} and \ac{AMO}
	operations based on \Cstd[11] Generics.
\\See Sections \ref{sec:rma}, \ref{sec:rma_nbi} and \ref{sec:amo}.
%
\item New nonblocking variants of remote memory access, \FUNC{SHMEM\_PUT\_NBI}
	and \FUNC{SHMEM\_GET\_NBI}.
\\See Sections \ref{subsec:shmem_put_nbi} and \ref{subsec:shmem_get_nbi}.
%
\item New atomic elemental read and write operations, \FUNC{SHMEM\_FETCH} and
	\FUNC{SHMEM\_SET}.
\\See Sections \ref{subsec:shmem_atomic_fetch} and \ref{subsec:shmem_atomic_set}
%
\item New alltoall data exchange operations, \FUNC{SHMEM\_ALLTOALL}
	and \FUNC{SHMEM\_ALLTOALLS}.
\\See Sections \ref{subsec:shmem_alltoall} and \ref{subsec:shmem_alltoalls}.
%
\item Added \CTYPE{volatile} to remotely accessible pointer argument in
	\FUNC{SHMEM\_WAIT} and \FUNC{SHMEM\_LOCK}.
\\See Sections \ref{subsec:shmem_wait_until} and \ref{subsec:shmem_lock}.
%
\item Deprecation of \FUNC{SHMEM\_CACHE}.
\\See Section \ref{subsec:shmem_cache}.
%
\end{itemize}




\section{Version 1.2}
Major changes in \openshmem[1.2] include
a new initialization routine (\FUNC{shmem\_init}),
improvements to the execution model with an explicit
library-finalization routine (\FUNC{shmem\_finalize}),
an early-exit routine (\FUNC{shmem\_global\_exit}),
namespace standardization,
and clarifications to several API descriptions.

The following list describes the specific changes in \openshmem[1.2]:
\begin{itemize}
%
\item Added specification of \VAR{pSync} initialization for all routines that use it.
%
\item Replaced all placeholder variable names \VAR{target} with \VAR{dest} to
      avoid confusion with \Fortran's \KEYWORD{target} keyword.
%
\item New Execution Model for exiting/finishing \openshmem programs.
\\See Section  \ref{subsec:execution_model}.
%
\item New library constants to support API that query version and name information.
\\See Section \ref{subsec:library_constants}.
%
\item New API \FUNC{shmem\_init} to provide mechanism to start an \openshmem
      program and replace deprecated \FUNC{start\_pes}.
\\See Section \ref{subsec:shmem_init}.
%
\item Deprecation of \FUNC{\_my\_pe} and \FUNC{\_num\_pes} routines.
\\See Sections \ref{subsec:shmem_my_pe} and \ref{subsec:shmem_n_pes}.
%
\item New API \FUNC{shmem\_finalize} to provide collective mechanism to cleanly
      exit an \openshmem program and release resources.
\\See Section \ref{subsec:shmem_finalize}.
%
\item New API \FUNC{shmem\_global\_exit} to provide mechanism to exit an
    \openshmem program.
\\See Section \ref{subsec:shmem_global_exit}.
%
\item Clarification related to the address of the referenced object in
    \FUNC{shmem\_ptr}.
\\See Section \ref{subsec:shmem_ptr}.
%
\item New API to query the version and name information.
\\See Section \ref{subsec:shmem_info_get_version} and \ref{subsec:shmem_info_get_name}.
%
\item \openshmem library API normalization. All \Cstd symmetric memory management
      API begins with  \FUNC{shmem\_}.
\\See Section \ref{subsec:shfree}.
%
\item Notes and clarifications added to \FUNC{shmem\_malloc}.
\\See Section \ref{subsec:shfree}.
%
\item Deprecation of \Fortran API routine \FUNC{SHMEM\_PUT}.
\\See Section \ref{subsec:shmem_put}.
%
\item Clarification related to \FUNC{shmem\_wait}.
\\See Section \ref{subsec:shmem_wait_until}.
%
\item Undefined behavior for null pointers without zero counts added.
\\See Annex \ref{sec:undefined}
%
\item Addition of new Annex for clearly specifying deprecated API and its
      support across versions of the \openshmem Specification.
\\See Annex \ref{sec:dep_api}.
%
\end{itemize}




\section{Version 1.1}
Major changes from \openshmem[1.0] to \openshmem[1.1] include
the introduction of the \HEADER{shmemx.h} header file for non-standard API
extensions,
clarifications to completion semantics and API descriptions in agreement with
the \ac{SGI} SHMEM specification,
and general readabilty and usability improvements to the document structure.

The following list describes the specific changes in \openshmem[1.1]:
\begin{itemize}
%
\item Clarifications of the completion semantics of memory synchronization
      interfaces.
\\See Section \ref{subsec:memory_order}.
%
\item Clarification of the completion semantics of memory load and store
      operations in context of \FUNC{shmem\_barrier\_all} and \FUNC{shmem\_barrier}
      routines.
\\See Section \ref{subsec:shmem_barrier_all} and \ref{subsec:shmem_barrier}.
%
\item Clarification of the completion and ordering semantics of
      \FUNC{shmem\_quiet} and \FUNC{shmem\_fence}.
\\See Section \ref{subsec:shmem_quiet} and \ref{subsec:shmem_fence}.
%
\item Clarifications of the completion semantics of \ac{RMA} and \ac{AMO}
      routines.
\\See Sections \ref{sec:rma} and \ref{sec:amo}
%
\item Clarifications of the memory model and the memory alignment requirements
      for symmetric data objects.
\\See Section \ref{subsec:memory_model}.
%
\item Clarification of the execution model and the definition of a \ac{PE}.
\\See Section \ref{subsec:execution_model}
%
\item Clarifications of the semantics of \FUNC{shmem\_pe\_accessible} and
      \FUNC{shmem\_addr\_accessible}.
\\See Section \ref{subsec:shmem_pe_accessible} and \ref{subsec:shmem_addr_accessible}.
%
\item Added an annex on interoperability with \ac{MPI}.
\\See Annex D.
%
\item Added examples to the different interfaces.
%
\item Clarification of the naming conventions for constant in \Cstd and
      \Fortran.
\\See Section \ref{subsec:library_constants} and \ref{subsec:shmem_wait_until}.
%
\item Added \ac{API} calls: \FUNC{shmem\_char\_p}, \FUNC{shmem\_char\_g}.
\\See Sections \ref{subsec:shmem_p} and \ref{subsec:shmem_g}.
%
\item Removed \ac{API} calls: \FUNC{shmem\_char\_put},
      \FUNC{shmem\_char\_get}.
\\See Sections \ref{subsec:shmem_put} and \ref{subsec:shmem_get}.
%
\item The usage of \CTYPE{ptrdiff\_t}, \CTYPE{size\_t}, and \CTYPE{int} in the
      interface signature was made consistent with the description.
\\See Sections \ref{subsec:coll}, \ref{subsec:shmem_iput}, and \ref{subsec:shmem_iget}.
%
\item Revised \FUNC{shmem\_barrier} example.
\\See Section \ref{subsec:shmem_barrier}.
%
\item Clarification of the initial value of \VAR{pSync} work arrays for
\FUNC{shmem\_barrier}.\\ See Section \ref{subsec:shmem_barrier}.
%
\item Clarification of the expected behavior when multiple \FUNC{start\_pes}
calls are encountered.
\\See Section \ref{subsec:start_pes}.
%
\item Corrected the definition of atomic increment operation.
\\See Section \ref{subsec:shmem_atomic_inc}.
%
\item Clarification of the size of the symmetric heap and when it is set.
\\See Section \ref{subsec:shfree}.
%
\item Clarification of the integer and real sizes for \Fortran \ac{API}.
\\See Sections \ref{subsec:shmem_atomic_add},
      \ref{subsec:shmem_atomic_compare_swap},
      \ref{subsec:shmem_atomic_swap},
      \ref{subsec:shmem_atomic_fetch_inc},
      \ref{subsec:shmem_atomic_inc}, and
      \ref{subsec:shmem_atomic_fetch_add}.
%
\item Clarification of the expected behavior on program \OPR{exit}.
\\See Section \ref{subsec:execution_model}, Execution Model.
%
\item More detailed description for the progress of \openshmem operations
provided.
\\See Section \ref{subsec:progress}.
%
\item Clarification of naming convention for non-standard interfaces and their
inclusion in \HEADER{shmemx.h}.
\\See Section \ref{subsec:bindings}.
%
\item Various fixes to \openshmem code examples across the Specification to
include appropriate header files.
%
\item Removing requirement that implementations should detect size mismatch and
return error information for \FUNC{shmalloc} and ensuring consistent
language.
\\See Sections \ref{subsec:shfree} and Annex \ref{sec:undefined}.
%
\item \Fortran programming fixes for examples.\\ See Sections
\ref{subsec:shmem_reductions} and \ref{subsec:shmem_wait_until}.
%
\item Clarifications of the reuse \VAR{pSync} and \VAR{pWork} across
collectives.
\\See Sections \ref{subsec:coll}, \ref{subsec:shmem_broadcast},
      \ref{subsec:shmem_collect} and \ref{subsec:shmem_reductions}.
%
\item Name changes for UV and ICE for \ac{SGI} systems.
\\See Annex \ref{sec:openshmem_history}.
%
\end{itemize}

} %end of setlength command that was started in frontmatter.tex



\end{document}

