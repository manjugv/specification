\apisummary{
    Creates a new set based on the input set and an array of triplets (start index, stride, size).
}

\begin{apidefinition}

\begin{Csynopsis}
int @\FuncDecl{shmem\_create\_ set\_strided\_multi}@(shmem\_set\_t parent\_set, int nstrides, int strides\_array[][3], shmem\_set\_t *new\_set);
\end{Csynopsis}

\begin{apiarguments}
     \apiargument{IN}{parent\_set}{A valid set. A valid set is either a pre-defined set or one created by application of one or more set creation operations.}
       \apiargument{IN}{nstrides}{Number of triplets}
       \apiargument{IN}{strides[][3]}{An array of stride triplets}
     \apiargument{OUT}{new\_set}{An opaque handle representing the new set.}
\end{apiarguments}

\apidescription{
    The shmem\_create\_set\_strided\_multi creates a new \ac{PE} set based on the array of \ac{PE} triplets (PE\_Start, PE\_stride and PE\_size) supplied to the routine. It is a local operation and can be called by any \ac{PE}. Upon return from a call to the routine, a handle to the new\_set is allocated to all calling \acp{PE}. If the \ac{PE} is a part of the new\_set, its index is a non-negative integer, otherwise it is negative. The new\_set contains subset of the \acp{PE} in the parent\_set specified by the triplet. The indices of the \acp{PE} in the new\_set are ordered by their indices in SHMEM\_SET\_WORLD.}

\apireturnvalues{
shmem\_create\_set\_strided\_multi  returns 0 upon success; otherwise, it returns a non-zero value.}

\apinotes{ None.}

\end{apidefinition}
