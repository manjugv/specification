A set is a group of \ac{PE} either created by the set creation operations defined in this section or is one of the predefined sets. The set creation operations are local operations and do not require communication between the \acp{PE} participating in the creation operation. The set is represented by an opaque object and cannot be passed from one \ac{PE} to another.

The \acp{PE} in the set are identified by a non-negative integer called the set index. The set indices are monotonically increasing and start from zero. Indexes are always assigned relative to the world order of the \ac{PE} number. If a \ac{PE} participates in the set creation operation but is not a part of the created set, the \ac{PE} index is set to a negative integer. This property is useful when querying the membership to a set.

The OpenSHMEM program initialized by either  \FUNC{shmem\_init} or  \FUNC{shmem\_init\_thread} creates \VAR{SET\_WORLD} during the initialization operation. This set can be referenced in the program by the constant \VAR{SHMEM\_SET\_WORLD}. Every \ac{PE} that calls  \FUNC{shmem\_init} or  \FUNC{shmem\_init\_thread} is a member of the \VAR{SHMEM\_SET\_WORLD} and its \ac{PE} index in \VAR{SHMEM\_SET\_WORLD} is the same as its global \ac{PE} number as returned by  \FUNC{shmem\_my\_pe}. The \VAR{SHMEM\_SET\_WORLD} is a valid set after return from the  \FUNC{shmem\_init} or  \FUNC{shmem\_init\_thread}. 
