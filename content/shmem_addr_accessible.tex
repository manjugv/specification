\apisummary{
    Determines whether an address is accessible via \openshmem data transfer
    routines from the specified remote \ac{PE}.
}

\begin{apidefinition}

\begin{Csynopsis}
int @\FuncDecl{shmem\_addr\_accessible}@(const void *addr, int pe);
\end{Csynopsis}

\begin{apiarguments}
    \apiargument{IN}{addr}{Local address of data object to query.}
    \apiargument{IN}{pe}{Integer id of a remote \ac{PE}.}
\end{apiarguments}

\apidescription{
    \FUNC{shmem\_addr\_accessible} is a query routine that indicates whether
    the address \VAR{addr} can be used to access the given data object on the
    specified \ac{PE} via \openshmem routines.
    
    This routine verifies that the data object is symmetric and accessible with
    respect to a remote \ac{PE} via \openshmem data transfer routines.  The
    specified address \VAR{addr} is the local address of the data object on the
    local \ac{PE}.
}

\apireturnvalues{
    The return value is \CONST{1} if the local address \VAR{addr} is also a symmetric
    address and the given data object is accessible via \openshmem routines on
    the specified remote \ac{PE}; otherwise, it is \CONST{0}.
}

\apinotes{
    None.
}

\end{apidefinition}
