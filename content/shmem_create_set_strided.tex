\apisummary{
    Creates a new set based on the input set and the triplet, which includes start index, size, and stride.
}

\begin{apidefinition}

\begin{Csynopsis}
int @\FuncDecl{shmem\_create\_set\_strided}@(shmem\_set\_t *parent\_set, int Index\_start, int Index\_stride, int size, shmem\_set\_t *new\_set);
\end{Csynopsis}

\begin{apiarguments}
     \apiargument{IN}{parent\_set}{A valid set. A valid set is either a pre-defined set or one created by application of one or more set creation operations.}
       \apiargument{IN}{index\_start}{The lowest PE index in the parent set, which will be part of new set.}
       \apiargument{IN}{PE\_index\_stride}{The integer displacement between the two consecutive PEs in the set.}
       \apiargument{IN}{PE\_size}{The number of PEs in the new set.}
     \apiargument{OUT}{new\_set}{An opaque handle representing the new set.}
\end{apiarguments}

\apidescription{
    The shmem\_create\_set\_strided creates a new PE set from the parent\_set based on the triplet (PE\_Start, PE\_stride and PE\_size) supplied to the routine. It is a local operation and can be called by any PE. Upon return from a call to the routine, a handle to the new\_set is allocated to all calling PEs. If the PE is a part of the new\_set, its index is a non-negative integer, otherwise it is negative. The new\_set contains the PE subset specified by the triplet. The indices of the PEs in the new\_set are ordered by their indices in SHMEM\_SET\_WORLD.
}

\apireturnvalues{
shmem\_create\_set\_strided  returns 0 upon success; otherwise, it returns a non-zero value.}

\apinotes{ The indices and \ac{PE} numbers are equivalent if the parent\_set is SHMEM\_SET\_WORLD.}

\end{apidefinition}
