\apisummary{
    Creates a new set that includes all \ac{PE} within the range provided as input range array.
}

\begin{apidefinition}
\begin{Csynopsis}
int @\FuncDecl{shmem\_create\_set\_range\_multi}@(shmem_set_t parent_set, int n, int ranges[][2], shmem_set_t *new_set)
\end{Csynopsis}

\begin{apiarguments}
     \apiargument{IN}{parent\_set}{A valid set. A valid set is either a pre-defined set or one created by application of one or more set creation operations.}
       \apiargument{IN}{nranges}{Number of ranges }
        \apiargument{IN}{ranges[][2]}{An array of range (low\_pe, high\_pe) tuples}
     \apiargument{OUT}{new\_set}{An opaque handle representing the new set.}
\end{apiarguments}

\apidescription{
   The shmem\_create\_set\_range\_multi creates a new \ac{PE} set that includes all \ac{PE} in the parent\_set, whose index is within the range specified by the array of ranges. It is a local operation and can be called by any \ac{PE}. This routine will return an opaque handle to the new\_set, which contains a subset of the \ac{PE} in the parent\_set specified by the ranges. If the calling \ac{PE} is a part of new\_set, its index in the new\_set is a valid non-negative integer, otherwise the \ac{PE} has a negative index. The indices of the \ac{PE} in the new\_set are ordered by their indices in SHMEM\_SET\_WORLD. }

\apireturnvalues{
shmem\_create\_set\_range\_multi  returns 0 upon success; otherwise, it returns a non-zero value.
}

\apinotes{ This routine could be used to build 2D and 3D sets.}

\end{apidefinition}